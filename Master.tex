\documentclass[czech,master]{diploma}
\usepackage[autostyle=true,czech=quotes]{csquotes}
\usepackage[backend=biber, style=iso-numeric, alldates=iso]{biblatex}
\usepackage{dcolumn}
\usepackage{subfig}
\usepackage{hyperref}
\usepackage{xurl}
\usepackage{tikz}
\usepackage[cpp]{diplomalst}

\ThesisAuthor{Bc. Filip Peterek}
\ThesisSupervisor{prof. RNDr. Marie Duží, CSc.}
\CzechThesisTitle{Implementace jazyka TIL-Script}
\EnglishThesisTitle{Implementation of the TIL-Script Language}
\SubmissionYear{2023}

\ThesisAssignmentFileName{spec.pdf}

\Acknowledgement{TODO: Doplnit poděkování, až bude práce hotová}

\CzechAbstract{
    Cílem práce je implementovat programovací jazyk TILScript. Jazyk TILScript slouží jako
    výpočetní varianta logického kalkulu TIL, jenž umožňuje jednoduchý strojový zápis konstrukcí
    Transparentní intenzionální logiky, ale také jejich následné provedení. Práce dále řeší
    praktické problémy s interpretací TILScriptu, a to například definice pojmenovaných funkcí,
    interakce s databází, apod. Dále se práce snaží navrhnout nadmnožinu jazyka TILSkript, která
    umožní konstrukce TILu nejen provádět, ale také analyzovat, vytvářet je, a pracovat s nimi.
}

\CzechKeywords{
    Transparentní intenzionální logika, TILScript, překladač
}

\EnglishAbstract{
    The goal of the thesis is the definition and implementation of the TILScript language.
    TILScript is a scripting language which serves the purpose of a computational variant of
    Transparent intensional logic, a logical calculus based on typed lambda calculi. TILScript
    allows for not just representation, but also execution of TIL constructions. This work also
    deals with practical problems of TILScript implementation, such as definitions of named
    functions, interaction with databases, and so on. Furthermore, this thesis attempts to define
    a superset of the TILScript language, which allows for not just the execution of constructions,
    but also for their creation and analysis.
}

\EnglishKeywords{
    Transparent intensional logic, TILScript, interpreter
}

\AddAcronym{TIL}{Transparentní intenzionální logika}
\AddAcronym{JVM}{Java Virtual Machine}

\addbibresource{biblatex-examples.bib}
\addbibresource{coffee.bib}

% Novy druh tabulkoveho sloupce, ve kterem jsou cisla zarovnana podle desetinne carky
\newcolumntype{d}[1]{D{,}{,}{#1}}


% Zacatek dokumentu
\begin{document}

\MakeTitlePages

% TODO
% Jsou v praci obrazky? Pokud ano vysazime jejich seznam a odstrankujeme.
% Pokud ne smazeme nasledujici dve makra.
\listoffigures
\clearpage

% TODO
% Jsou v praci tabulky? Pokud ano vysazime jejich seznam a odstrankujeme.
% Pokud ne smazeme nasledujici dve makra.
\listoftables
\clearpage

% A nasleduje text zaverecne prace.
% \chapter{Úvod}
\label{sec:Introduction}

% TODO: Ocitovat n-gramove modely, word2vec, ChatGPT
% GPT-3: https://arxiv.org/abs/2005.14165

Analýza přirozeného jazyka jako disciplína stále rychleji stoupá na oblibě i důležitosti. Jistě
málokomu unikly například n-gramové modely založené na predikci následujícího slova
z předcházejících $n$ slov, či vektorové modely jako Word2Vec, umožňující reprezentovat význam
slov pomocí vektorů. Poslední dobou se velmi často mluví o jazykovém modelu GPT-3.

Výčet přístupů k analýze přirozeného jazyka však nekončí n-gramy a neuronovými sítěmi.

\endinput

\chapter{Transparentní intenzionální logika}
\label{sec:TILIntroduction}

% TODO: Citace
Transparentní intenzionální logika (TIL) je logický systém založený na typovaném lambda kalkulu.
TIL je využíván k logické analýze přirozeného jazyka. Oproti tradičnímu lambda kalkulu, jenž
se využívá jako komputační model, tedy jako pouhý prostředek k dosažení konkrétní hodnoty --
výsledku, v Transparentní intenzionální logice hraje konstrukce kalkulu často důležitější roli,
než hodnota, kterou by konstrukce po provedení zkonstruovala.

Jako příklad využití lambda kalkulu jako výpočetní model lze uvést např. funkcionální programovací
jazyk Haskell. Interně je Haskell kompilován do lambda kalkulu (přesněji do jeho supersetu
obsahujícího např. čísla nebo logické hodnoty, která jinak v lambda kalkulu musíme kódovat pomocí
Churchova kódování -- K-kombinátorů, apod.). Ultimátně v Haskellu ovšem lambda kalkul slouží pouze
k získání konkrétního výsledku. Nadefinujeme vztah mezi vstupem a výstupem, a program napsaný
v Haskellu nám vstup transformuje. Pokud zanedbáme efektivitu programu, nezajímá nás, jakým
způsobem program spočítal výsledek, dokud jej spočítal správně.

Naopak Transparentní intenzionální logika je hyperintenzionálním kalkulem, který nám umožňuje
vytvářet konstrukce vypovídající o jiných konstrukcích. TIL vychází z myšlenky, že výraz
přirozeného jazyka sice označuje denotát -- konkrétní individuum, významem výrazu ovšem není
samotný denotát, který ani nemusí nutně existovat. Význam výrazu je abstraktní a lze jej zachytit
konstrukcí. Daná konstrukce poté při provedení zkonstruuje denotát výrazu. Jako příklad lze uvést
například výraz "francouzský král." V době psaní této práce Francie krále nemá. Výraz nemá žádný
denotát, neuvádí žádné konkrétní individuum. Přesto výrazu "francouzský král" rozumíme, výraz má
svůj význam, jen v současné době neuvádí žádnou osobu. A budeme-li chtít o významu výrazu
"francouzský král" něco vypovědět, například že francouzský král je monarchou v čele Francie,
daný monarcha nemusí existovat. Dále lze uvést například rozdíl mezi výrazy "logaritmus 1024
o základě 2" a "5 + 5". Denotátem obou výrazů je 10. Zadáme-li do interpreteru Haskellu výrazy
\lstset{language=Haskell}
\lstinline{logBase 2 1024} a \lstinline{5 + 5}, získáme v obou případech stejný výsledek.
V přirozeném jazyce ovšem chápeme značný rozdíl mezi oběma výrazy, ačkoliv mají stejný denotát.
"Logaritmus 1024 o základě 2" vyjadřuje číslo, kterým musíme umocnit dvojku, abychom získali 1024.
Výraz "5 + 5" očividně vyjadřuje úplně jinou matematickou operaci a jeho výsledek spočítáme jiným
postupem.

\begin{figure}
    \centering
    \begin{tikzpicture}
        \node[draw, fit={(0, 0) (2, 1)},              xshift=3cm, inner sep=0pt, label=center:Výraz] (A) {};
        \node[draw, fit={(0, 0) (2, 1)}, yshift=-5cm,             inner sep=0pt, label=center:Konstrukce] (B) {};
        \node[draw, fit={(0, 0) (2, 1)}, yshift=-5cm, xshift=6cm, inner sep=0pt, label=center:Denotát] (C) {};

        \path (A) -- node[sloped] (ab) {vyjadřuje}  (B);
        \path (A) -- node[sloped] (ac) {označuje}   (C);
        \path (B) -- node[sloped] (bc) {konstruuje} (C);

        \draw [-latex]          (A)--(ab)--(B);
        \draw [-latex] [dashed] (A)--(ac)--(C);
        \draw [-latex]          (B)--(bc)--(C);
    \end{tikzpicture}
    \caption{Schéma procedurální sémantiky TIL}
    \label{fig:til-semantics}
\end{figure}

Denotátem výrazu může být jak objekt z báze, konstrukce, i funkce.

Jak již bylo zmíněno, Transparentní intenzionální logika vychází z typovaného lambda kalkulu, proto
také každý objekt musí mít svůj typ. Pro správné pochopení TILu, a tedy i této práce, je tak nutné 
znát typovou hierarchii TIL.

\subsection{Báze}

Báze je kolekce vzájemně disjunktních neprázdných množin, které dohromady definují universum
diskurzu. Tyto množiny definují atomické objekty. Každá množina dále objektům určuje určitá
základní kritéria (např. pokud jako jednu z množin báze zvolíme množinu $\mathbb{N}$, víme, že
všechny objekty z této množiny budou čísla).

\subsection{Typy 1. řádu}


\subsection{Rozvětvěná hierarchie typů}

\subsection{Konstrukce TIL}

\subsection{Charakteristické rysy TIL}

\endinput

% \input{Chapters/SampleChapter1.tex}
% \input{Chapters/SampleChapter2.tex}
% \input{Chapters/TechnicalDetails.tex}
% \chapter{Závěr}

Cíl práce -- vytvořit funkční překladač jazyka TIL-Script, byl splněn. Byl implementován prototyp
překladače, který dokáže překládat programy jazyka TIL-Script, chybí mu ovšem některé v praxi
potřebné optimalizace (například optimalizace koncového volání). Nedostatky jsou však v textu
zdokumentovány, aby bylo možné překladač dále rozšiřovat a vylepšovat.

Dále práce rozšířila jazyk TIL-Script o nové prvky -- jako jsou například definice nových typů,
komentáře, výrazy \lstinline{Import}, nebo textové řetězce.

Implementována byla také standardní knihovna pro základní práci s jazykem TIL-Script, nebo
matematická knihovna definující několik užitečných matematických funkcí.

Překladač umožňuje implementovat TIL-Script funkce, které interně volají funkce jazyka Java -- tím
se jazyku TIL-Script otvírá také celý Java ekosystém, včetně knihoven pro jazyk Java.

Při tvorbě překladače bylo myšleno také na budoucí rozvoj. Současnou implementaci překladač je možné
nahradit implementací novou. Pokud bude nový překladač implementovat potřebná rozhraní, bude tento
překladač plně kompatibilní se standardní a matematickou knihovnou, případně s jakoukoliv jinou
knihovnou psanou pro jazyk TIL-Script.

\endinput


% Seznam literatury
\printbibliography[title={Literatura}, heading=bibintoc]

% Prilohy
\appendix
\input{Chapters/Appendix1.tex}
\chapter{Ukázky zdrojových kódů}

\lstinputlisting[caption={Ukázka definice intenzí.}]{SourceCodes/Intensions.tils}

\clearpage

\lstinputlisting[caption={Ukázka implementace registrátoru v jazyce Kotlin},language=Kotlin]{SourceCodes/Registrar.kt}

\clearpage

\lstinputlisting[caption={Ukázka implementace TIL-Script funkce v jazyce Kotlin},language=Kotlin]{SourceCodes/Function.kt}

\clearpage

\endinput


% Priloha vlozena primo do hlavniho LaTeX souboru. Ne vsechny prilohy je nutne mit ve zvlastnich souborech.
\chapter{Dlouhý zdrojový kód}
\lstinputlisting[label=src:CppExternal,caption={Dlouhý zdrojový kód v jazyce C++ načtený s externího souboru}]{SourceCodes/ArraySortingAlgorithms.cpp}

\end{document}
