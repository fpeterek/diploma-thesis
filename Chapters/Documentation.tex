\chapter{Uživatelská dokumentace}

Tato kapitola je věnována uživatelské dokumentaci. Součástí dokumentace je návod, jak spustit
TIL-Script program, popis standardní i matematické knihovny, návod na implementaci vlastní knihovny,
a nakonec také ukázka jednoduchého TIL-Script programu.

\section{Překlad programu}

Překladač byl psán pro platformu Java, proto pro spuštění překladače jazyka TIL-Script musíme mít
nainstalované Java prostředí (JRE). Máme-li JRE nainstalované, překladač můžeme spustit ručně, nebo
pomocí přiloženého pomocného skriptu.

Překladač spouštíme vždy z příkazové řádky, neboť pro něj momentálně neexistuje grafické rozhraní.

Při ručním spuštění je třeba manuálně spustit Java prostředí a specifikovat JAR soubor obsahující
kód TIL-Script překladače. Překladači je potřeba předat jako argument název souborů, které chceme
přeložit. Pokud je interpret TIL-Scriptu jediný Java archiv, který načítáme, není třeba specifikovat
tzv. \textit{Main Class}, tedy třídu obsahující statickou metodu \lstinline{void main()} (neboť
specifikace této třídy je součástí souboru \lstinline{manifest} obsaženém v archivu).

\begin{lstlisting}[caption={Spuštění překladače}]
java -jar tilscript.jar script.tils
\end{lstlisting}

Pokud chceme kromě překladače načíst také TIL-Script knihovny, musíme uvést nejen všechny archivy,
jenž potřebuje Java prostředí načíst, ale také hlavní třídu.

\begin{lstlisting}[caption={Spuštění překladače s načtením knihoven}]
java -cp tilscript.jar:libs/math.jar org.fpeterek.tilscript.interpreter.MainKt script.tils
\end{lstlisting}

\subsection{tilscript.sh}

Nejjednodušší způsob, jak překladač jazyka TIL-Script spustit, je využít pomocný skript
\lstinline{tilscript.sh}. Skript \lstinline{tilscript.sh} využívá pouze funkcionalitu definovanou
standardem POSIX, proto by tento skript měl fungovat korektně na všech operačních systémech
splňujících standard POSIX. Dále se standardu POSIX musí držet také shell, který bude tento pomocný
skript interpretovat\footnote{Můžeme tedy používat například ZSH nebo Bash. Naopak shell Fish není
kompatibilní se standardem POSIX, proto skript nebude fungovat korektně}.

Skript \lstinline{tilscript.sh} předpokládá, že se nachází ve stejné složce jako soubor
\lstinline{tilscript.jar}, tedy archiv obsahující přeložený kód překladače. Dále tento skript
předpokládá existenci adresáře \lstinline{libs/}, opět ve stejné složce, jako skript samotný.
Skript při spuštění automaticky načte všechny Java archivy ve složce \lstinline{libs/}, spustí
Java prostředí, zajistí načtení všech knihoven i TIL-Script překladače a korektně uvede hlavní
třídu překladače. Všechny argumenty, které skript obdrží, poté automaticky předá TIL-Script
překladači.

\begin{lstlisting}[caption={Spuštění překladače za využití pomocného skriptu}]
./tilscript.sh script.tils
\end{lstlisting}

\section{Standardní knihovna}

Standardní knihovna jazyka TIL-Script obsahuje základní funkce pro práci s objekty Transparentní
intenzionální logiky. Dále obsahuje definice atomických typů a tří proměnných. Z důvodu náročnosti
implementace některým funkcím chybí implementace, proto je můžeme pouze zmínit, nemůžeme však
provést jejich aplikaci.

Nakonec je třeba uvést, že současný stav nemusí reprezentovat také konečný stav standardní knihovny.
Na základě zpětné vazby uživatelů lze standardní knihovnu v budoucnu rozšiřovat.

\subsection{Funkce}

\subsubsection{Deklarace}

Zde uvedeným funkcím chybí implementace z důvodu její náročnosti a časové složitosti. Představme si
například všeobecný kvantifikátor. Logickou pravdivost řady tvrzení či úsudků lze dokázat například
pomocí důkazových kalkulů (rezoluční metoda, přirozená dedukce). Tyto metody dokazování jsou ovšem
čistě syntaktické, nedokáží tedy dokázat pravdivost tvrzení jako například $\forall x[P(x)]$. V takových
případech bychom se museli uchýlit k sémantické analýze predikátu $P$.
Predikát ovšem může být netriviální a jeho analýza velmi složitá. Iterace přes celý obor hodnot
predikátu je naopak nepraktická nebo nemožná. Pokud by predikát $P$ byla funkce typu $(o\tau)$
(nezapomeňme, že v Transparentní intenzionální logice pracujeme pouze s funkcemi), iterovat přes
obor hodnot by bylo nemožné, neboť množina reálných čísel je nespočetná. Množina všech validních
hodnot 64-bitového čísla s plovoucí řádovou čárkou spočetná je, proto je možné přes tato čísla
iterovat, není to však praktické, neboť by výpočet nemusel v skončit v rozumném čase.

\subsubsection*{Funkce: \lstinline|ForAll|}
Typ: \lstinline|(Bool (Bool Any1))|

Všeobecný kvantifikátor

\subsubsection*{Funkce: \lstinline|Exist|}
Typ: \lstinline|(Bool (Bool Any1))|

Existenční kvantifikátor

\subsubsection*{Funkce: \lstinline|Sing|}
Typ: \lstinline|(Bool (Bool Any1))|

Singularizátor

\subsubsection*{Funkce: \lstinline|Every|}
Typ: \lstinline|((Bool (Bool Any1)) (Bool Any1))|

Omezený kvantifikátor

\subsubsection*{Funkce: \lstinline|Some|}
Typ: \lstinline|((Bool (Bool Any1)) (Bool Any1))|

Omezený kvantifikátor

\subsubsection*{Funkce: \lstinline|No|}
Typ: \lstinline|((Bool (Bool Any1)) (Bool Any1))|

Omezený kvantifikátor

\subsubsection*{Funkce: \lstinline|Sub|}
Typ: \lstinline|(Construction Construction Construction Construction)|

Funkce \textit{Sub} substituční metody

\subsubsection*{Funkce: \lstinline|TrueC|}
Typ: \lstinline|(Bool Construction)|

Třída konstrukcí pravdivých ve všech valuacích $v$.

\subsubsection*{Funkce: \lstinline|FalseC|}
Typ: \lstinline|(Bool Construction)|

Třída konstrukcí nepravdivých ve všech valuacích $v$.

\subsubsection*{Funkce: \lstinline|ImproperC|}
Typ: \lstinline|(Bool Construction)|

Třída konstrukcí $v$-nevlastních pro všechny valuace $v$.

\subsubsection*{Funkce: \lstinline|TrueC|}
Typ: \lstinline|(Bool ((Bool Time) World))|

Třída propozic pravdivých ve všech valuacích $v$.

\subsubsection*{Funkce: \lstinline|FalseC|}
Typ: \lstinline|(Bool ((Bool Time) World))|

Třída propozic nepravdivých ve všech valuacích $v$.

\subsubsection*{Funkce: \lstinline|TrueC|}
Typ: \lstinline|(Bool ((Bool Time) World))|

Třída propozic $v$-nevlastních ve všech valuacích $v$.

\subsubsection{Definice}

Následující funkce již jsou korektně definovány a lze provést jejich aplikaci na argumenty. Ke každé
funkci je přiložena ukázka jejího využití.

\subsubsection*{Funkce: \lstinline|ListOf|}

\lstinline{ListOf} slouží pouze jako syntaktický cukr pro tvorbu seznamů. \lstinline{ListOf}
aplikujeme na alespoň jeden argument, počet argumentů je ale shora neomezený. Jediným omezením je,
že všechny argumenty musí být stejného typu. Parser jazyka TIL-Script poté aplikaci
\lstinline{ListOf} převede na korektní sestavení seznamu pomocí funkce \lstinline{ListOfOne} a
funkcí \lstinline{Cons}.


\begin{lstlisting}[caption={Ukázka využití ListOf}]
-- Následující dvě konstrukce jsou ekvivalentní
['ListOf '1 '2 '3 '4].
['Cons '1 ['Cons '2 ['Cons '3 ['ListOfOne '4]]]].
\end{lstlisting}

\subsubsection*{Funkce: \lstinline|ListOfOne|}
Typ: \lstinline{(List(Any1) Any1)}

Funkce \lstinline{ListOfOne} vytvoří seznam obsahující jediný prvek.

\begin{lstlisting}[caption={Ukázka využití ListOfOne}]
['ListOfOne 'Pi].
\end{lstlisting}

\subsubsection*{Funkce: \lstinline|Cons|}
Typ: \lstinline{(List(Any1) Any1 List(Any1))}

Funkce \lstinline{Cons} vytvoří nový seznam vložením prvku na začátek již existujícího seznamu.
Jelikož jsou seznamy definovány induktivně, a zároveň jsou neměnné, není třeba již existující seznam
kopírovat. Proto lze tuto operaci provést v konstantním čase.

\begin{lstlisting}[caption={Ukázka využití Cons}]
['Cons 'Pi ['LisOf '1 '2 '3]].
\end{lstlisting}

\subsubsection*{Funkce: \lstinline|Head|}
Typ: \lstinline{(Any1 List(Any1))}

Funkce \lstinline{Head} vrátí první prvek seznamu. Seznam musí být neprázdný, v opačném případě
funkce nevrací nic (vrací \lstinline{Nil}).

\begin{lstlisting}[caption={Ukázka využití Head}]
['Head ['ListOf '1 '2 '3]]. -- Konstruuje 1
\end{lstlisting}

\subsubsection*{Funkce: \lstinline|Tail|}
Typ: \lstinline{(List(Any1) List(Any1))}

Funkce \lstinline{Tail} vrátí seznam bez jeho prvního prvku. Funkce je nedefinovaná pro prázdné
seznamy.

\begin{lstlisting}[caption={Ukázka využití Head}]
['Tail ['ListOf '1 '2 '3]]. -- Konstruuje seznam [2, 3]
\end{lstlisting}

\subsubsection*{Funkce: \lstinline|IsEmpty|}
Typ: \lstinline{(Bool List(Any1))}

Aplikací \lstinline{IsEmpty} na prázdný seznam získáme hodnotu \lstinline{True}. Aplikací
na neprázdný seznam získáme \lstinline{False}.

\begin{lstlisting}[caption={Ukázka využití IsEmpty}]
['IsEmpty ['ListOf '1 '2 '3]]. -- False
['IsEmpty ['Tail ['ListOfOne '1]]]. -- True
\end{lstlisting}

\subsubsection*{Funkce: \lstinline|EmptyListOf|}
Typ: \lstinline{(List(Any1) Type)}

Funkce \lstinline{EmptyListOf} jako svůj jediný vstup přijímá objekt typu \lstinline{Type}.
Výsledkem aplikace na typ je poté prázdný seznam objektů specifikovaného typu.

\begin{lstlisting}[caption={Ukázka využití EmptyListOf}]
['IsEmptyOf 'Int].
\end{lstlisting}

\subsubsection*{Funkce: \lstinline|Print|, \lstinline{Println}}
Typ: \lstinline{(Any1 Any1)}

Funkce \lstinline{Print}, \lstinline{Println} zajistí výpis svého argumentu na standardní výstup
programu a poté vrátí svůj jediný argument. Funkce \lstinline{Println} vypíše také oddělovač řádků
(v systému GNU/Linux znak LF, v systému Windows sekvenci CRLF). Funkce \lstinline{Print} tento
oddělovač nevypisuje.

\begin{lstlisting}[caption={Ukázka využití Print, Println}]
['Println '['+ '1 '2]]. -- Vypíše konstrukci ['+ '1 '2]
['Print ['+ '1 '2]]. -- Vypíše konstrukci ['+ '1 '2]
\end{lstlisting}

\subsubsection*{Funkce: \lstinline{If}}
Typ: \lstinline{(Any1 Bool Any1 Any1)}

Funkce \lstinline{If} je, narozdíl od všech ostatních funkcí, prováděna líně. Funkce \lstinline{If}
přijímá tři argumenty. První argument je objekt typu \lstinline{Bool}. Druhým argumentem je hodnota,
kterou funkce vrátí, je-li první argument \lstinline{True}. Jinak funkce \lstinline{If} vrátí svůj
druhý argument. Konstrukce, která konstruuje argument funkce \lstinline{If}, je ovšem provedena až
poté, co překladač zkontroluje hodnotu prvního argumentu, aby nedošlo ke zbytečnému provedení
konstrukce a následnému zahození výsledku.

\begin{lstlisting}[caption={Ukázka využití If}]
['If ['> x y]
    ['Println '"x is greater than y"]
    ['Println '"y is greater than or equal to x"]].
\end{lstlisting}

\subsubsection*{Funkce: \lstinline{Cond}}

\lstinline{Cond} slouží pouze jako syntaktický cukr pro funkci \lstinline{If}. \lstinline{Cond} tedy
není funkcí sama o sobě, překladačem je ovšem při procesu tvorby AST přeložena na sérii vnořených
aplikací funkce \lstinline{If}.

Máme-li pouze jednu podmínku se dvěma větvemi, \textit{if} a \textit{else}, je využití funkce
\lstinline{If} jednoduché a čitelné. Máme-li však větví více, musíme aplikace funkce \lstinline{If}
zanořit -- větví \textit{else} tak musí být další aplikace \lstinline{If}.

Syntaktický cukr \lstinline{Cond} nám umožňuje zjednodušit zápis funkce \lstinline{If} a vyhnout
se zanořování. \lstinline{Cond} přijímá nespecifikovaný, avšak sudý počet argumentů. Lichým
argumentem je vždy podmínka. Sudým argumentem je poté konstrukce, která se provede, je-li podmínka
pravdivá. Překladač během překladu přeloží aplikaci \lstinline{Cond} na zanořené aplikace funkce
\lstinline{If}. Podmínky se tedy nevyhodnocují všechny najednou, ale jedna po druhé, dokud není
nalezena první pravdivá podmínka. Není-li pravdivá ani jedna podmínka, \lstinline{Cond} vrací
\lstinline{Nil}.

\begin{lstlisting}[caption={Ukázka využití Cond}]
['Cond
    ['= x '2] ['Log2 y]
    ['= x '10] ['Log10 y]
    'True ['Log y]]. -- catch-all podmínka
\end{lstlisting}

\subsubsection*{Funkce: \lstinline{Progn}}
Typ: \lstinline{(Any2 Any1 Any2)}

Funkce \lstinline{Progn} přijímá dva argumenty. První argument ignoruje, druhý argument vrátí, jak
jej dostala. Funkce \lstinline{Progn} je tedy ekvivalentem funkce \textit{False} lambda kalkulu
($\lambda x \lambda y . y$, případně $\lambda x, y: y$ v TIL). Díky principu kompozicionality je aplikace
\lstinline{Progn} $v$-nevlastní, neobdrží-li první argument. V takovém případě tedy vůbec nedojde
k vyhodnocení druhého argumentu.

Funkci \lstinline{Progn} využijeme například při výpisu např. do souboru nebo na standardní výstup.
Pro analýzu přirozeného jazyka většinou není potřeba, kódovat přirozená čísla pomocí Churchova
kódování v Transparentní intenzionální logice nepotřebujeme.

Funkce \lstinline{Progn} je binární funkcí, existuje pro ni ovšem podobný syntaktický cukr, jako pro
\lstinline{Cond} nebo \lstinline{ListOf}. Předáme-li funkci \lstinline{Progn} více než dva
argumenty, překladač aplikaci \lstinline{Progn} na více argumentů rozepíše na vnořené aplikace
binární funkce \lstinline{Progn}. Narozdíl od \lstinline{Cond}, \lstinline{Progn} je tedy
skutečnou funkcí.

Název \lstinline{Progn} je převzat z jazyka Lisp.

\begin{lstlisting}[caption={Ukázka využití Progn}]
-- Vypíše 'x + y = ' a součet x a y
-- Vrátí součet x+y, protože Println vrátí svůj argument
['Progn
    ['Print '"x + y = "]
    ['Println ['+ x y]]].
\end{lstlisting}

\subsubsection*{Funkce: \lstinline{Tr}}
Typ: \lstinline{(Construction Any1)}

Trivializuje svůj argument.

\begin{lstlisting}[caption={Ukázka využití Tr}]
['Tr ['+ '1 '2]]. -- Kompozice konstruuje '3
\end{lstlisting}

\subsubsection*{Funkce: \lstinline{TypeOf}}
Typ: \lstinline{(Type Any1)}

Vrátí typ svého argumentu. Může být užitečné například v typově polymorfních funkcích,
potřebujeme-li provést rozhodnutí na základě typu argumentu.

\begin{lstlisting}[caption={Ukázka využití TypeOf}]
-- Volba funkce na základě typu argumentu typově polymorfní funkce
\x: Any1 -> Any1 [['If ['= ['TypeOf x] 'Int] 'DiscreteLog 'Log10] x].
\end{lstlisting}

\subsubsection*{Funkce: \lstinline{IsNil}}
Typ: \lstinline{(Bool Any1)}

Vrátí \lstinline{True}, neobdrží-li žádný argument (tedy je-li argumentem \lstinline{Nil}). Tato
funkce porušuje princip kompozicionality, je ovšem potřeba např. k ošetření chyb.

\begin{lstlisting}[caption={Ukázka využití IsNil}]
x -> Real = ['/ a b].
['If ['IsNil x]
    -- Vypíšeme chybu a ukončíme program
    ['Progn
        ['Println '"Program obdržel nesprávný vstup"]
        ['Exit '-1]]
    -- Program obdržel validní vstup, hodnota 1 bude ignorována
    '1].
\end{lstlisting}

\subsubsection*{Funkce: \lstinline{Exit}}

Aplikací \lstinline{Exit} ukončíme překlad programu. Argument funkce určuje návratovou hodnotu
programu.

\begin{lstlisting}[caption={Ukázka využití Exit}]
x -> Real = ['/ a b].
['If ['IsNil x]
    -- Vypíšeme chybu a ukončíme program
    ['Progn
        ['Println '"Program obdržel nesprávný vstup"]
        ['Exit '-1]]
    -- Program obdržel validní vstup, hodnota 1 bude ignorována
    '1].
\end{lstlisting}

\subsubsection*{Funkce: \lstinline{RandomInt}}
Typ: \lstinline{(Int DeviceState)}

Funkce \lstinline{RandomInt} vrací náhodné celé číslo v intervalu $\bigl \langle 0; 2^{64}-1 \bigr \rangle$.

\begin{lstlisting}[caption={Ukázka využití RandomInt}]
['RandomInt deviceState].
\end{lstlisting}

\subsubsection*{Funkce: \lstinline{Random}}
Typ: \lstinline{(Real DeviceState)}

Funkce \lstinline{Random} vrací náhodné reálné číslo v intervalu $\bigl \langle 0; 1 \bigr \rangle$.

\begin{lstlisting}[caption={Ukázka využití Random}]
['* ['Random deviceState] '100]. -- náhodné číslo v rozsahu 0-100
\end{lstlisting}

\subsubsection*{Funkce: \lstinline{Random}}
Typ: \lstinline{(Real DeviceState)}

Funkce \lstinline{Random} vrací náhodné reálné číslo v intervalu $\bigl \langle 0; 1 \bigr \rangle$.

\begin{lstlisting}[caption={Ukázka využití Random}]
['* ['Random deviceState] '100]. -- náhodné číslo v rozsahu 0-100
\end{lstlisting}

\subsubsection*{Funkce: \lstinline{Not}}
Typ: \lstinline{(Bool Bool)}

Logická negace.

\begin{lstlisting}[caption={Ukázka využití Not}]
['Not ['= x y]].
\end{lstlisting}

\subsubsection*{Funkce: \lstinline{And}}
Typ: \lstinline{(Bool Bool Bool)}

Logická konjunkce.

\begin{lstlisting}[caption={Ukázka využití And}]
[And ['= x y] ['= x z]].
\end{lstlisting}

\subsubsection*{Funkce: \lstinline{Or}}
Typ: \lstinline{(Bool Bool Bool)}

Logická disjunkce.

\begin{lstlisting}[caption={Ukázka využití Or}]
[Or ['= x y] ['= x z]].
\end{lstlisting}

\subsubsection*{Funkce: \lstinline{Implies}}
Typ: \lstinline{(Bool Bool Bool)}

Implikace.

\begin{lstlisting}[caption={Ukázka využití Implies}]
['Implies ['And ['= x y] ['= x z]] ['= y z]].
\end{lstlisting}

\subsubsection*{Funkce: \lstinline{OneTuple}}
Typ: \lstinline{(Tuple(Any1) Any1)}

Vytvoří n-tici obsahující právě jeden prvek -- obdržený argument. Lze využít například při rekurzivní
tvorbě n-tice.

\begin{lstlisting}[caption={Ukázka využití OneTuple}]
['OneTuple '1].
\end{lstlisting}

\subsubsection*{Funkce: \lstinline{MkTuple}}

\lstinline{MkTuple} slouží jako syntaktický cukr pro tvorbu n-tic. \lstinline{MkTuple} přijímá
nespecifikovaný počet argumentů, překladačem je poté přeložena na aplikaci \lstinline{OneTuple}
a následné vložení prvků na začátek n-tice.

\begin{lstlisting}[caption={Ukázka využití OneTuple}]
['MkTuple '1 '3.5 'True].
\end{lstlisting}

\subsubsection*{Funkce: \lstinline{PrependToTuple}}
Typ: \lstinline{(Any1 Any2 Any3)}

\lstinline{PrependToTuple} přijímá dva argumenty. Prvním argumentem je libovolná hodnota. Druhý
argument musí být vždy n-tice. \lstinline{PrependToTuple} vytvoří novou (n+1) vložením prvního
argumentu na začátek n-tice specifikované druhým argumentem.

\begin{lstlisting}[caption={Ukázka využití PrependToTuple}]
['PrependToTuple '4 ['MkTuple '1 '3.5 'True]]. -- Vytvoří n-tici (4, 1, 3.5, True)
\end{lstlisting}

\subsubsection*{Funkce: \lstinline{Get}}
Typ: \lstinline{(Any1 Any2 Int)}

Funkce \lstinline{Get} umožňuje získat prvek n-tice na požadované pozici. Prvním argumentem je
n-tice, druhým je index prvku. Prvky indexujeme od nuly. Funkce \lstinline{Get} je nedefinovaná,
je-li index větší nebo roven velikosti n-tice.

\begin{lstlisting}[caption={Ukázka využití Get}]
['Get '1 ['MkTuple 'E 'Pi]]. -- Pi
\end{lstlisting}

\subsubsection*{Funkce: \lstinline{TupleLen}}
Typ: \lstinline{(Int Any1)}

Funkce \lstinline{TupleLen} vrátí délku n-tice. Využití funkce je převážně v typově polymorfních
funkcích, neboť není-li funkce typově polymorfní, museli jsme typ explicitně uvést, a tedy délku
n-tice známe. Argumentem musí být n-tice, funkce je typově polymorfní proto, aby argumentem mohly
být n-tice libovolné délky.

\begin{lstlisting}[caption={Ukázka využití TupleLen}]
-- vynásobíme první prvek n-tice délkou n-tice
[\x: Any1 -> Int ['* ['TupleLen x] ['Get x '0]]].
\end{lstlisting}

\subsubsection*{Funkce: \lstinline{TupleLen}}
Typ: \lstinline{(Int Any1)}

Funkce \lstinline{TupleLen} vrátí délku n-tice. Využití funkce je převážně v typově polymorfních
funkcích, neboť není-li funkce typově polymorfní, museli jsme typ explicitně uvést, a tedy délku
n-tice známe. Argumentem musí být n-tice, funkce je typově polymorfní proto, aby argumentem mohly
být n-tice libovolné délky.

\begin{lstlisting}[caption={Ukázka využití TupleLen}]
-- vynásobíme první prvek n-tice délkou n-tice
[\x: Any1 -> Int ['* ['TupleLen x] ['Get x '0]]].
\end{lstlisting}

\subsubsection*{Funkce: \lstinline{Char}}
Typ: \lstinline{(Text Text Int)}

Funkce \lstinline{Char} vrátí znak textového řetězce na požadované pozici. Prvky indexujeme od nuly.
Funkce \lstinline{Char} je nedefinovaná, je-li index větší nebo roven velikosti řetězce.


\begin{lstlisting}[caption={Ukázka využití Char}]
['Char '"TIL" '1]. -- "I"
\end{lstlisting}

\subsubsection*{Funkce: \lstinline{CatS}}
Typ: \lstinline{(Text Text Text)}

Funkce \lstinline{CatS} spojí dva textové řetězce.

\begin{lstlisting}[caption={Ukázka využití Char}]
-- Kompozice konstruuje řetězec "Transparentní intenzionální logika"
['CatS '"Transparentní intenzionální" '" logika"].
\end{lstlisting}

\subsubsection*{Funkce: \lstinline{HeadS}}
Typ: \lstinline{(Text Text)}

Funkce \lstinline{HeadS} vrátí první znak řetězce. Funkce je nedefinovaná pro prázdné řetězce.

\begin{lstlisting}[caption={Ukázka využití HeadS}]
['HeadS '"TIL"]. -- "T"
\end{lstlisting}

\subsubsection*{Funkce: \lstinline{TailS}}
Typ: \lstinline{(Text Text)}

Funkce \lstinline{HeadS} vrátí řetězec bez prvního znaku. Funkce je nedefinovaná pro prázdné
řetězce.

\begin{lstlisting}[caption={Ukázka využití TailS}]
[TailS '"TIL"]. -- "IL"
\end{lstlisting}

\subsubsection*{Funkce: \lstinline{LenS}}
Typ: \lstinline{(Int Text)}

Funkce \lstinline{LenS} vrátí délku textového řetězce.

\begin{lstlisting}[caption={Ukázka využití LenS}]
[LenS '"TIL"]. -- 3
\end{lstlisting}

\subsubsection*{Funkce: \lstinline{IsBefore}, \lstinline{IsBeforeOrEq}, \lstinline{IsAfter},
  \lstinline{IsAfterOrEq}}
Typ: \lstinline{(Bool Time)}

Funkce slouží k porovnávání časových okamžiků. Pro obyčejnou rovnost lze využít funkci
\lstinline{=}.

\begin{lstlisting}[caption={Ukázka porovnávání časových okamžiků}]
['IsBefore t1 t2].
\end{lstlisting}

\subsubsection*{Funkce: \lstinline{Now}}
Typ: \lstinline{(Time DeviceState)}

Funkce \lstinline{Now} vrátí aktuální systémový čas.

\begin{lstlisting}[caption={Ukázka využití Now}]
[['PrezidentCR ['Now deviceState]] w].
\end{lstlisting}

\subsubsection*{Funkce: \lstinline{IntToText},
\lstinline{RealToText},
\lstinline{IntToTime},
\lstinline{TextToInt},
\lstinline{TextToReal},
\lstinline{TimeToInt},
\lstinline{IntToReal},
\lstinline{RealToInt}
}

Funkce slouží k převodu mezi typy. Při převodu z reálných čísel na čísla celá dochází k ořezání
čísla (zaokrouhlení směrem k nule).

\begin{lstlisting}[caption={Ukázka využití konverzí}]
['Log ['IntToReal ['+ '2 '3]]].
['CatS '"Log 10 = " ['RealToText ['Log '10]]].
\end{lstlisting}

\subsubsection*{Funkce:
\lstinline{IsVariable},
\lstinline{IsComposition},
\lstinline{IsClosure},
\lstinline{IsExecution},
\lstinline{IsFunction},
\lstinline{IsTrivialization},
\lstinline{IsSymbol},
\lstinline{IsList},
\lstinline{IsValue},
\lstinline{IsTuple},
\lstinline{IsConstruction},
\lstinline{IsStruct}
}

Typ: \lstinline{(Bool Any1)}

Tyto funkce umožňují ověřit, zda je argument konkrétní hodnotou (\lstinline{IsValue}), symbolickou
hodnotou (\lstinline{IsSymbol}), seznamem, n-ticí, strukturou nebo konstrukcí. Také nám umožňují
určit, o jaký typ konstrukce se jedná.

\begin{lstlisting}[caption={Ukázka využití IsSymbol}]
[\x: Int -> Int
    ['If ['IsSymbol x]
        Nil -- neumíme umocnit symbolickou hodnotu
        ['* x x]]]. -- konkrétní hodnotu umocníme
\end{lstlisting}

\subsubsection*{Funkce: \lstinline{NilAt}}\label{nilat-fn}
Typ: \lstinline{(Int Text Tuple(Int Int Text))}

Funkce \lstinline{NilAt} je nedefinována na všech argumentech. Funkce umožňuje zkonstruovat hodnotu
\lstinline{Nil}. \lstinline{Nil} jde také zmínit přímo, \lstinline{NilAt} nám ovšem umožňuje vnitřní
reprezentaci \lstinline{Nil} obohatit také o důvod, proč byla konstrukce $v$-nevlastní, a pozici,
kde byla zavolána funkce, jež \lstinline{Nil} vrátila. Tímto způsobem můžeme dosáhnout lepšího
hlášení chyb.

Argumentem funkce je důvod, proč byla vrácena hodnota \lstinline{Nil}, a pozice, kde došlo
k aplikaci funkci na argumenty, na kterých není definována. Často zde budeme chtít použít proměnnou
\lstinline{callsite}, viz \peteref{callsite-var}.

Ukázka je schválně jednoduchá, logaritmus v matematické knihovně tuto situaci řeší sám, v praxi tedy
není třeba kontrolovat argument logaritmu manuálně. Proměnná \lstinline{callsite} konstruuje
pozici ve zdrojovém kódu, kde došlo k aplikaci funkce konstruované uzávěrem.

\begin{lstlisting}[caption={Ukázka využití NilAt}]
[\x: Real -> Real
    ['If ['Not ['> x 0.0]]
        ['NilAt '"Argument logaritmu musí být kladný" 'callsite]
        ['Log x]]].
\end{lstlisting}

%TODO: Continue if necessary

\subsection{Typy}

Standardní knihovna jazyka TIL-Script definuje atomické typy popsané v kapitole
\peteref{tilscript-chapter}, tedy typy
\lstinline{Bool}, \lstinline{Type}, \lstinline{Int}, \lstinline{Text}, \lstinline{Indiv},
\lstinline{Real}, \lstinline{Time}, \lstinline{World}, \lstinline{DeviceState}.

Dále definuje seznamy a n-tice. Žádné struktury standardní knihovna nedefinuje.

\subsection{Proměnné}

Standardní knihovna definuje pouze tři proměnné.

\subsubsection{\lstinline{w -> World}}

Proměnná \lstinline{w} označuje současný svět. Objekty typu \lstinline{World} nemají žádnou
inherentní hodnotu, využívají se hlavně k označení intenzí. Proměnná \lstinline{w} proto existuje,
aby si ji uživatel nemusel zbytečně vytvářet sám.

\subsubsection{\lstinline{deviceState -> DeviceState}}

Proměnná \lstinline{deviceState} existuje ze stejného důvodu, jako proměnná \lstinline{w}. Objekty
typu \lstinline{DeviceState} nemají žádnou skutečnou hodnotu a jsou využívány pro označení funkcí,
jejichž hodnota závisí na stavu zařízení, na němž je program spuštěn.

\subsubsection{\lstinline{callsite -> Tuple(Text Int Int)}}\label{callsite-var}

Proměnná \lstinline{callsite} je automaticky vytvářená proměnná. Proměnná \lstinline{callsite}
je vytvořena při aplikaci funkce, a obsahuje pozici ve zdrojovém kódu, kde byla aplikace funkce
provedena. Proměnná slouží především k umožnění lepšího hlášení chyb.

Proměnná slouží převážně jako argument funkce \lstinline{NilAt}, viz \peteref{nilat-fn}.

\section{Matematická knihovna}

\section{Implementace knihovny}

\section{Ukázka TIL-Script programu}

\endinput
