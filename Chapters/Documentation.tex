\chapter{Uživatelská dokumentace}

Tato kapitola je věnována uživatelské dokumentaci. Součástí dokumentace je návod, jak spustit
TIL-Script program, popis standardní i matematické knihovny, návod na implementaci vlastní knihovny,
a nakonec také ukázka jednoduchého TIL-Script programu.

\section{Překlad programu}

Překladač byl psán pro platformu Java, proto pro spuštění překladače jazyka TIL-Script musíme mít
nainstalované Java prostředí (JRE). Máme-li JRE nainstalované, překladač můžeme spustit ručně, nebo
pomocí přiloženého pomocného skriptu.

Překladač spouštíme vždy z příkazové řádky, neboť pro něj momentálně neexistuje grafické rozhraní.

Při ručním spuštění je třeba manuálně spustit Java prostředí a specifikovat JAR soubor obsahující
kód TIL-Script překladače. Překladači je potřeba předat jako argument název souborů, které chceme
přeložit. Pokud je interpret TIL-Scriptu jediný Java archiv, který načítáme, není třeba specifikovat
tzv. \textit{Main Class}, tedy třídu obsahující statickou metodu \lstinline{void main()} (neboť
specifikace této třídy je součástí souboru \lstinline{manifest} obsaženém v archivu).

\begin{lstlisting}[caption={Spuštění překladače}]
java -jar tilscript.jar script.tils
\end{lstlisting}

Pokud chceme kromě překladače načíst také TIL-Script knihovny, musíme uvést nejen všechny archivy,
jenž potřebuje Java prostředí načíst, ale také hlavní třídu.

\begin{lstlisting}[caption={Spuštění překladače s načtením knihoven}]
java -cp tilscript.jar:libs/math.jar org.fpeterek.tilscript.interpreter.MainKt script.tils
\end{lstlisting}

\subsection{tilscript.sh}

Nejjednodušší způsob, jak překladač jazyka TIL-Script spustit, je využít pomocný skript
\lstinline{tilscript.sh}. Skript \lstinline{tilscript.sh} využívá pouze funkcionalitu definovanou
standardem POSIX, proto by tento skript měl fungovat korektně na všech operačních systémech
splňujících standard POSIX. Dále se standardu POSIX musí držet také shell, který bude tento pomocný
skript interpretovat\footnote{Můžeme tedy používat například ZSH nebo Bash. Naopak shell Fish není
kompatibilní se standardem POSIX, proto skript nebude fungovat korektně}.

Skript \lstinline{tilscript.sh} předpokládá, že se nachází ve stejné složce jako soubor
\lstinline{tilscript.jar}, tedy archiv obsahující přeložený kód překladače. Dále tento skript
předpokládá existenci adresáře \lstinline{libs/}, opět ve stejné složce, jako skript samotný.
Skript při spuštění automaticky načte všechny Java archivy ve složce \lstinline{libs/}, spustí
Java prostředí, zajistí načtení všech knihoven i TIL-Script překladače a korektně uvede hlavní
třídu překladače. Všechny argumenty, které skript obdrží, poté automaticky předá TIL-Script
překladači.

\begin{lstlisting}[caption={Spuštění překladače za využití pomocného skriptu}]
./tilscript.sh script.tils
\end{lstlisting}

\section{Standardní knihovna}

Standardní knihovna jazyka TIL-Script obsahuje základní funkce pro práci s objekty Transparentní
intenzionální logiky. Dále obsahuje definice atomických typů a tří proměnných. Z důvodu náročnosti
implementace některým funkcím chybí implementace, proto je můžeme pouze zmínit, nemůžeme však
provést jejich aplikaci.

Nakonec je třeba uvést, že současný stav nemusí reprezentovat také konečný stav standardní knihovny.
Na základě zpětné vazby uživatelů lze standardní knihovnu v budoucnu rozšiřovat.

\subsection{Funkce}

\subsubsection{Deklarace}

%TODO: Reformat
\begin{table}
    \begin{tabular}{ l l l }
        \hline
        Funkce  &  Typ  &  Popis \\
        \hline
        \lstinline|ForAll| & \lstinline|(Bool (Bool Any1))| & Všeobecný kvantifikátor \\
        \lstinline|Exist|  & \lstinline|(Bool (Bool Any1))| & Existenční kvantifikátor \\
        \lstinline|Sing|   & \lstinline|(Bool (Bool Any1))| & Singularizátor \\
        \lstinline|Every|  & \lstinline|((Bool (Bool Any1)) (Bool Any1))| & Omezený kvantifikátor \\
        \lstinline|Some|   & \lstinline|((Bool (Bool Any1)) (Bool Any1))| & Omezený kvantifikátor \\
        \lstinline|No|     & \lstinline|((Bool (Bool Any1)) (Bool Any1))| & Omezený kvantifikátor \\
        \lstinline|Sub|    & \lstinline|(Construction Construction Construction Construction)| &
        Funkce \textit|Sub| substituční metody \\
        \lstinline|TrueC|  & \lstinline|(Bool Construction)| & Pravdivost hodnoty konstruované konstrukcí \\
        \lstinline|FalseC| & \lstinline|(Bool Construction)| & Nepravdivost hodnoty konstruované konstrucí \\
        \lstinline|ImproperC|     & \lstinline|(Bool Construction)| & Je konstrukce $v$-nevlastní \\
        \lstinline|TrueC|  & \lstinline|(Bool ((Bool Time) World))| & Pravdivost propozice \\
        \lstinline|FalseC| & \lstinline|(Bool ((Bool Time) World))| & Nepravdivost propozice \\
        \lstinline|TrueC|  & \lstinline|(Bool ((Bool Time) World))| & Je propozice $v$-nevlastní \\
    \end{tabular}
\end{table}

\subsection{Typy}

\subsection{Proměnné}

\section{Matematická knihovna}

\section{Implementace knihovny}

\section{Ukázka TIL-Script programu}

\endinput
