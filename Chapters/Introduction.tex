\chapter{Úvod}
\label{sec:Introduction}

% TODO: Ocitovat n-gramove modely, word2vec, GPT-3
% W2V: https://arxiv.org/pdf/1301.3781.pdf
% GPT-3: https://arxiv.org/abs/2005.14165

Analýza přirozeného jazyka jako disciplína stále rychleji stoupá na oblibě i důležitosti. Jistě
málokomu unikly například n-gramové modely založené na predikci následujícího slova
z předcházejících $n$ slov, či vektorové modely jako Word2Vec, umožňující reprezentovat význam
slov pomocí vektorů. Poslední dobou se velmi často mluví o jazykovém modelu GPT-3.

Výčet přístupů k analýze přirozeného jazyka však nekončí n-gramy a neuronovými sítěmi. K analýze
přirozených jazyků lze přistoupit také pomocí logiky. Mezi průkopníky tohoto přístupu patřil také
český logik Pavel Tichý, tvůrce Transparentní intenzionální logiky.

Transparentní intenzionální logika (dále také TIL) je logický systém založený na typovaném lambda
kalkulu. TIL umožňuje modelovat a analyzovat výrazy přirozeného jazyka za pomocí logiky. Jako
doplněk TILu poté vznikl také funkcionální programovací jazyk TIL-Script, sloužící k interpretaci
konstrukcí Transparentní intenzionální logiky. Syntax i sémantika jazyka TIL-Script jsou silně
inspirovány TILem, ovšem přirozeně s určitými úpravami tak, aby byl TIL-Script rozumně zapisovatelný
a interpretovatelný na počítači.

A právě interpretací jazyka TIL-Script se zabývá tento text. Součástí práce tak je navázání
na předcházející vývoj jazyka TIL-Script, rozšíření gramatiky nutné pro akomodaci nových
prvků jazyka (např. výraz \textit{Import} sloužící k importu symbolů definovaných v jiném
souboru), nebo také samotná interpretace jazyka TIL-Script, společně s kontrolou typové koherence.
Dále práce navrhuje způsob, jak volat funkce platformy JVM. Tento přístup otevírá jazyku TIL-Script
celý ekosystém platformy JVM. Práce tedy nijak neřeší a neimplementuje získávání informací
z již existující databáze, soustředí se spíše na zjednodušení přístupu k existujícím knihovnám, aby
do budoucna nebyl problém rychle a jednoduše naprogramovat přístup k libovolnému zdroji dat.

Nakonec se práce snaží navrhnout určitou nadmnožinu jazyka TIL-Script tak, aby TIL-Script mohl
sloužit nejen k interpretaci konstrukcí TIL, ale také k jejich tvorbě a analýze. V současné době
pro analýzu a práci s konstrukcemi jazyka TIL-Script existují nástroje psané v jazyce Java.
Existuje-li však v současné době programovací jazyk vytvořený přímo pro logiky pracující s TILem,
není důvod tento jazyk nevyužít a nerozšířit také o nástroje pro analýzu, tvorbu i transformaci
TIL-Script konstrukcí. Pro pokročilejší práci s jazykem TIL-Script by tak nebylo třeba učit se jiný
programovací jazyk (Java), nebo využívat externí nástroje, ale naopak by byl k dispozici již
familiérní nástroj, jehož základy jsou již příznivcům Transparentní intenzionální logiky dobře
známy. Cílem této nadmnožiny ovšem není předefinovat TIL, či nějak výrazně měnit jádro jazyka
TIL-Script, jinak by se ostatně také nemohlo jednat o nadmnožinu. Tyto nové prvky jsou od jádra
jazyka TIL-Script, které slouží převážně jako výpočetní varianta TIL, jednoduše oddělitelné. Při
práci s překladačem, jehož implementace je součástí této práce, není problém se této nadmnožině
vyhnout a nevyužívat ji. Také lze, na základě tohoto textu, implementovat překladač jazyka
TIL-Script bez jakýchkoliv navrhovaných rozšíření.

Práce se do znatelné míry inspiruje výzkumem amerického profesora Johna McCarthyho a programovacím
jazykem Lisp. Jelikož Lisp vychází z lambda kalkulu, pracuje s parciálními funkcemi, a nerozlišuje
mezi programem a daty (tedy Lisp programy můžou pracovat s, generovat a transformovat Lisp
konstrukce), lze mezi Lispem a TILem nalézt řadu podobností. Primárním odlišovacím znakem
Transparentní intenzionální logiky je ovšem rigorózně definovaný typový systém.

Text samotný je rozdělen do čtyř částí. V první části budou představeny základy Transparentní
intenzionální logiky, jejichž znalost je nutná k pochopení práce. Druhá část představí programovací
jazyk TIL-Script a vyznačí změny a úpravy, které tato práce navrhuje. Třetí část popíše technické
řešení a implementaci překladače. V poslední části poté bude zdokumentována standardní i
matematická knihovna, které jsou součástí implementace. Dále bude poslední část obsahovat návod
na použití, návod na implementaci vlastní knihovny, a nakonec také ukázky programů psaných v jazyce
TIL-Script.

V terminologii programovacích jazyků a překladačů existuje spousta zavedených anglických názvů,
jejichž český ekvivalent je méně zavedený, případně neexistuje vůbec. Proto bude práce ve spoustě
případů uvádět také anglický ekvivalent k českému výrazu.

\endinput
