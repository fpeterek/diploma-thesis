\chapter{Transparentní intenzionální logika}
\label{sec:TILIntroduction}

% TODO: Citace
Transparentní intenzionální logika (TIL) je logický systém založený na typovaném lambda kalkulu.
TIL je využíván k logické analýze přirozeného jazyka. Oproti tradičnímu lambda kalkulu, jenž
se využívá jako komputační model, tedy jako pouhý prostředek k dosažení konkrétní hodnoty --
výsledku, v Transparentní intenzionální logice hraje konstrukce kalkulu často důležitější roli,
než hodnota, kterou by konstrukce po provedení zkonstruovala.

Jako příklad využití lambda kalkulu jako výpočetní model lze uvést např. funkcionální programovací
jazyk Haskell. Interně je Haskell kompilován do lambda kalkulu (přesněji do jeho supersetu
obsahujícího např. čísla nebo logické hodnoty, která jinak v lambda kalkulu musíme kódovat pomocí
Churchova kódování -- K-kombinátorů, apod.). Ultimátně v Haskellu ovšem lambda kalkul slouží pouze
k získání konkrétního výsledku. Nadefinujeme vztah mezi vstupem a výstupem, a program napsaný
v Haskellu nám vstup transformuje. Pokud zanedbáme efektivitu programu, nezajímá nás, jakým
způsobem program spočítal výsledek, dokud jej spočítal správně.

Naopak Transparentní intenzionální logika je hyperintenzionálním kalkulem, který nám umožňuje
vytvářet konstrukce vypovídající o jiných konstrukcích. TIL vychází z myšlenky, že výraz
přirozeného jazyka sice označuje denotát -- konkrétní individuum, významem výrazu ovšem není
samotný denotát, který ani nemusí nutně existovat. Význam výrazu je abstraktní a lze jej zachytit
konstrukcí. Daná konstrukce poté při provedení zkonstruuje denotát výrazu. Jako příklad lze uvést
například výraz "francouzský král." V době psaní této práce Francie krále nemá. Výraz nemá žádný
denotát, neuvádí žádné konkrétní individuum. Přesto výrazu "francouzský král" rozumíme, výraz má
svůj význam, jen v současné době neuvádí žádnou osobu. A budeme-li chtít o významu výrazu
"francouzský král" něco vypovědět, například že francouzský král je monarchou v čele Francie,
daný monarcha nemusí existovat. Dále lze uvést například rozdíl mezi výrazy "logaritmus 1024
o základě 2" a "5 + 5". Denotátem obou výrazů je 10. Zadáme-li do interpreteru Haskellu výrazy
\lstset{language=Haskell}
\lstinline{logBase 2 1024} a \lstinline{5 + 5}, získáme v obou případech stejný výsledek.
V přirozeném jazyce ovšem chápeme značný rozdíl mezi oběma výrazy, ačkoliv mají stejný denotát.
"Logaritmus 1024 o základě 2" vyjadřuje číslo, kterým musíme umocnit dvojku, abychom získali 1024.
Výraz "5 + 5" očividně vyjadřuje úplně jinou matematickou operaci a jeho výsledek spočítáme jiným
postupem.

\begin{figure}
    \centering
    \begin{tikzpicture}
        \node[draw, fit={(0, 0) (2, 1)},              xshift=3cm, inner sep=0pt, label=center:Výraz] (A) {};
        \node[draw, fit={(0, 0) (2, 1)}, yshift=-5cm,             inner sep=0pt, label=center:Konstrukce] (B) {};
        \node[draw, fit={(0, 0) (2, 1)}, yshift=-5cm, xshift=6cm, inner sep=0pt, label=center:Denotát] (C) {};

        \path (A) -- node[sloped] (ab) {vyjadřuje}  (B);
        \path (A) -- node[sloped] (ac) {označuje}   (C);
        \path (B) -- node[sloped] (bc) {konstruuje} (C);

        \draw [-latex]          (A)--(ab)--(B);
        \draw [-latex] [dashed] (A)--(ac)--(C);
        \draw [-latex]          (B)--(bc)--(C);
    \end{tikzpicture}
    \caption{Schéma procedurální sémantiky TIL}
    \label{fig:til-semantics}
\end{figure}

Denotátem výrazu může být nejen objekt z báze, ale i konstrukce nebo funkce.

Jak již bylo zmíněno, Transparentní intenzionální logika vychází z typovaného lambda kalkulu, proto
také každý objekt musí mít svůj typ. Pro správné pochopení TILu, a tedy i této práce, je tak nutné 
znát typovou hierarchii TIL.

\section{Báze}

Báze je kolekce vzájemně disjunktních neprázdných množin, které dohromady definují s jakými objekty
budeme pracovat. Tyto množiny definují atomické objekty. Každá množina dále objektům určuje určitá
základní kritéria (např. pokud jako jednu z množin báze zvolíme množinu $\mathbb{N}$, víme, že
všechny objekty z této množiny budou čísla). Vždy se ovšem jedná pouze o nejnutnější
a nejzákladnější vlastnosti. Báze například nemá žádný vliv na vlastnosti proměnlivé v čase.

Bázi volíme dle potřeb konkrétní aplikace a univerza diskurzu. Například používáme-li TIL k logické
analýze matematických vět, jako bázi lze zvolit například množinu celých čísel, množinu reálných
čísel, a množinu pravdivostních hodnot. Musíme však vzít v potaz, že tato báze neobsahuje čísla
komplexní.

Patří-li objekt x do množiny $\alpha$ z báze, říkáme, že se jedná o objekt typu $\alpha$.
K explicitnímu uvedení typu objektu \textit{x} využíváme zápis $x/\alpha$. Množinám tvořícím bázi
lze přirozeně říkat typy.

Pro analýzu přirozeného jazyka se většinou volí objektová báze skládající se z typů {$o$, $\iota$,
$\tau$, $\omega$}. Tyto typy jsou podrobněji popsány v tabulce \ref{tab:default-base}.

\begin{table}
    \caption{Výchozí báze pro analýzu přirozeného jazyka}
    \label{tab:default-base}
    \centering

    \begin{tabular} { | l l | }
        \hline
        Typ      & Popis typu \\
        \hline
        $o$      & Množina pravdivostních hodnot \\
        $\iota$  & Množina individuí (univerzum diskurzu) \\
        $\tau$   & Množina časových okamžiků/reálných čísel \\
        $\omega$ & Množina logicky možných světů \\
        \hline
    \end{tabular}
\end{table}

\section{Funkce}

V matematice se jako základní molekulární typ využívají relace. Funkce je poté speciální typ relace,
který je zprava jednoznačný. V TIL je však základním molekulárním typem funkce. Chceme-li v TIL
vyjádřit $n$-ární relaci nad množinou $\alpha_1 \times ... \times \alpha_n$, lze tak samozřejmě
udělat definicí $n$-ární funkce z $\alpha_1 \times ... \times \alpha_n$ do $o$, která každému prvku
z $\alpha_1 \times ... \times \alpha_n$ přiřadí pravdivostní hodnotu na základě toho, zda prvek
do relace patří, nebo ne.

Narozdíl od tradičního lambda kalkulu, kde jsou funkce pouze nulární nebo unární, v Transparentní
intenzionální logice není arita funkce omezena. Dále můžeme v TIL pracovat s funkcemi parciálními.

\subsection{Intenze a extenze}

V TIL dále rozlišujeme funkce na tzv. \textit{intenze} a \textit{extenze}. Intenze jsou funkce
z možných světů. Extenze jsou funkce, jejichž doménou množina možných světů není, a tudíž jejich
funkční hodnota nezávisí na stavu světa.

Intenze jsou obecně funkce typu $(\alpha\omega)$ pro libovolný typ $\alpha$. Nejčastěji se však
jedná o funkce typu $((\alpha\tau)\omega)$, tedy funkce zobrazující možné světy do chronologií
objektů typu $\alpha$.

\section{Konstrukce TIL}

Konstrukce v Transparentní intenzionální logice jsou abstraktní procedury. Tyto procedury jsou
strukturované -- nejedná se o množiny, mají pevně danou strukturu, a na uspořádání případných
podprocedur záleží. Tyto konstrukce lze podle definovaných pravidel provést. Provedením konstrukce,
získáme výstup, případně nezískáme nic. Konstrukce, které nekonstruují žádný výstup, nazýváme
\textit{nevlastní} (anglicky \textit{improper}). V TIL pracujeme s šesti druhy konstrukcí.

Jak již bylo zmíněno, konstrukce můžou v TIL operovat nejen nad objekty, které nejsou konstrukcemi,
tedy nad objekty z báze, ale také nad jinými konstrukce. Konstrukce však může operovat pouze nad
konstrukcemi nižšího řádu, než je konstrukce samotná, viz \ref{type-order}. Každou podkonstrukci,
kterou musíme provést při provedení konstrukce, nazýváme \textit{konstituentem}. V TIL existuje
šest různých druhů konstrukcí. Dvě atomické -- mají pouze jeden konstituent, a to sebe samotné,
a čtyři molekulární. Atomickými konstrukcemi jsou \textit{trivializace} a \textit{proměnné}. Mezi
molekulární konstrukce poté řadíme \textit{kompozice}, \textit{uzávěry}, \textit{provedení} a
\textit{dvojí provedení}.

\textit{Proměnné} jsou konstrukce, které na základě valuace \textit{v} \textit{v}-konstruují
objekty. Skutečnost, že proměnná $x$ \textit{v}-konstruuje hodnotu typu $\alpha$ značíme
$x \rightarrow_v \alpha$.

\lstset{language=Lisp}
\textit{Trivializace} pro libovolný objekt \textit{X} konstruuje samotný objekt \textit{X}.
Konstrukce je v Transparentní intenzionální logice potřebná, neboť výchozím módem pro konstrukce
je provedení. Samotná konstrukce \textit{X} by tak byla automaticky provedena, a místo konstrukce
\textit{X} bychom dostali pouze její denotát. Pokud bychom chtěli zkonstruovat konstrukci
\textit{X}, musíme ji trivializovat. Tím se provede pouze konstrukce trivializace. A protože
trivializace nemá jiný konstituent, než sebe samotnou, konstrukce \textit{X} se tak neprovede.
V literatuře se trivializace \textit{X} tradičně značí ${}^0X$. Obzvlášť se používá také zápis
$'X$. Tento zápis je poté využit i v jazyce TILScript. Trivializaci lze považovat za ekvivalent
funkce \lstinline{QUOTE} z jazyka Lisp. Trivializace taktéž bývá využívána ke konstruování hodnot,
které nelze provést (objekty z báze, funkce) a tudíž je nelze zmínit netrivializované.

\textit{Kompozice} je využívána k aplikaci funkcí. Kompozice $[X Y_1...Y_m]$ značí aplikaci funkce
konstruované konstrukcí $X$ na argumenty zkonstruované konstrukcemi $Y_1,...,Y_m$. Pokud konstrukce
$X$ konstruuje funkci $f$, všechny podkonstrukce $Y_1,...,Y_m$ konstruují hodnotu, a je-li funkce
$f$ na daných argumentech definovaná, kompozice $v$-konstruuje funkční hodnotu na těchto
argumentech. V opačném případě je kompozice nevlastní.

\textit{Uzávěr} $\lambda x_1...x_m Y$ je konstrukce $v$-konstruující funkci. $x_1,...x_m$ musí
být navzájem různé proměnné, $Y$ musí být konstrukcí. Konstruce uzávěru je velmi podobná abstrakci
v lambda kalkulu. Narozdíl od lambda kalkulu však v TILu může uzávěr konstruovat funkce s aritou
vyšší než jedna. Uzávěr nemůže být nikdy nevlastní, může však konstruovat tzv.
\textit{degenerovanou funkci}, tedy funkci, která je nedefinovaná na celém definičním oboru.

\textit{Provedení} ${}^1X$ je konstrukce $v$-konstruující objekt konstruovaný konstrukcí $X$.
Pokud je konstrukce $X$ $v$-nevlastní, je provedení ${}^1X$ také $v$-nevlastní. Jelikož je však
provedení výchozím módem pro objekty, většinou se neuvádí. Provést lze pouze konstrukce. Objekty
z báze (tedy čísla, individua, apod...) či funkce nelze provést, jejich provedení nekonstruuje nic.
Proto je potřeba tyto objekty vždy trivializovat.

\textit{Dvojí provedení} ${}^2X$ je poslední z výčtu konstrukcí. Je-li $X$ konstrukcí
$v$-konstruující konstrukci $Y$, a $v$-konstruuje-li konstrukce $Y$ objekt $Z$, pak ${}^2X$
$v$-konstruuje $Z$. V opačném případě je dvojí provedení ${}^2X$ $v$-nevlastní.

Jiné konstrukce v Transparentní intenzionální logice neexistují.

\subsection{Princip kompozicionality} 

Princip kompozicionality je důležitým rysem Transparentní intenzionální logiky. Princip
kompozicionality uvádí, že je-li libovolný konstituent konstrukce $X$ $v$-nevlastní a v dané
valuaci nekonstruuje žádnou hodnotu, pak je $v$-nevlastní i konstrukce $X$.

\section{Typy 1. řádu} \label{fst-order}

% TODO: Doplnit citaci
Definice je skoro slovo od slova převzata z knihy
\textit{TIL jako procedurální logika -- Průvodce zvídavého čtenáře Transparentní intensionální
logikou}. Tato sekce slouží jako krátké vysvětlení základů Transparentní intenzionální logiky
se čtenář může obrátit například na tuto knihu.

Nechť \textit{B} je báze. Pak:

\begin{enumerate}[i)]
    \item Každá množina z báze \textit{B} je atomický typ řádu 1 nad \textit{B}.
    \item Nechť $\alpha, \beta_1, ...,\beta_m (m > 0)$ jsou typy řádu 1 nad \textit{B}. Pak soubor
        všech \textit{m}-árních parciálních funkcí $(\alpha\beta_1...\beta_m)$, tedy zobrazení z 
        $\beta_1 \times ... \times \beta_m$ do $\alpha$, je molekulární typ řádu 1 nad \textit{B}.
    \item Nic jiného není typem řádu 1 nad bází \textit{B}.
\end{enumerate}

\section{Rozvětvěná hierarchie typů} \label{type-order}

%TODO: Doplnit citaci
Definice je opět skoro slovo od slova převzata z.

Nechť \textit{B} je báze. Pak:

\subsection{$T_1$ (typy řádu 1)}
Viz sekce \ref{fst-order}.

\subsection{$C_n$ (konstrukce řádu n)}

\begin{enumerate}[i)]
    \item Nechť $x$ je proměnná $v$-konstruující objekt typu řádu $n$. Pak $x$ je
        \textit{kontrukce řádu n nad b}.
    \item Nechť $X$ je prvek typu řádu $n$. Pak trivializace ${}^0X$, provedení ${}^1X$ a dvojí
        provedení ${}^2X$ jsou \textit{konstrukcemi řádu n nad b}.
    \item Nechť $X, Y_1, ... Y_m$ jsou konstrukce řádu $n$ nad \textit{B}. Pak kompozice 
        $[X Y_1...Y_m]$ je \textit{konstrukce řádu n nad b}.
    \item Nechť $x_1,...,x_m$ jsou vzájemně různé proměnné a $X$ je konstrukce řádu $n$
        nad \textit{B}. Pak uzávěr $[\lambda x_1 ... x_m X]$ je \textit{konstrukce řádu n nad b}.
    \item Nic jiného není konstrukcí řádu $n$ nad bází \textit{B}.
\end{enumerate}

\subsection{$T_{n+1}$ (typy řádu n+1)}

Nechť $*_n$ je kolekce všech konstrukcí řádu $n$ nad $B$.

\begin{enumerate}[i)]
    \item $*_n$ a každý typ řádu $n$ jsou \textit{typy řádu n+1 nad B}.
    \item Jsou-li $\alpha, \beta_1,...,\beta_m$ typy řádu $n+1$ nad \textit{B}, pak
        $(\alpha \beta_1...\beta_m)$, tedy kolekce parciálních funkcí, je
        \textit{typy řádu n+1 nad B}.
    \item Nic jiného není typ řádu $n+1$ nad \textit{B}.
\end{enumerate}

% \section{Charakteristické rysy TIL}

% \subsection{Princip kompozicionality}

% Princip kompozicionality říká, že význam výrazu je jednoznačně určen významy jeho podsložek.
% Z principu kompozicionality také vyplývá, že nemá-li konstrukce žádný význam (tedy jedná se o
% nevlastní konstrukci), nemají význam ani konstrukce, pro něž je daná nevlastní konstrukce
% konstituentem.

% \subsection{Antikontextualismus}

% Antikontextualismus znamená, že význam výrazu je stejný nezávisle na stavu světa.

% \subsection{Antiaktualismus}

\section{Analytické a empirické výrazy}

V TIL pracujeme s dvěma typy výrazů.

\endinput
