\chapter{Transparentní intenzionální logika}
\label{sec:TILIntroduction}

% TODO: Citace
Transparentní intenzionální logika (TIL) je logický systém založený na typovaném lambda kalkulu.
TIL je využíván k logické analýze přirozeného jazyka. Oproti tradičnímu lambda kalkulu, jenž
se využívá jako komputační model, tedy jako pouhý prostředek k dosažení konkrétní hodnoty --
výsledku, v Transparentní intenzionální logice hraje konstrukce kalkulu často důležitější roli,
než hodnota, kterou by konstrukce po provedení zkonstruovala.

Jako příklad využití lambda kalkulu jako výpočetní model lze uvést např. funkcionální programovací
jazyk Haskell. Interně je Haskell kompilován do lambda kalkulu (přesněji do jeho supersetu
obsahujícího např. čísla nebo logické hodnoty, která jinak v lambda kalkulu musíme kódovat pomocí
Churchova kódování -- K-kombinátorů, apod.). Ultimátně v Haskellu ovšem lambda kalkul slouží pouze
k získání konkrétního výsledku. Nadefinujeme vztah mezi vstupem a výstupem, a program napsaný
v Haskellu nám vstup transformuje. Pokud zanedbáme efektivitu programu, nezajímá nás, jakým
způsobem program spočítal výsledek, dokud jej spočítal správně.

Naopak Transparentní intenzionální logika je hyperintenzionálním kalkulem, který nám umožňuje
vytvářet konstrukce vypovídající o jiných konstrukcích. TIL vychází z myšlenky, že výraz
přirozeného jazyka sice označuje denotát -- konkrétní individuum, významem výrazu ovšem není
samotný denotát, který ani nemusí nutně existovat. Význam výrazu je abstraktní a lze jej zachytit
konstrukcí. Daná konstrukce poté při provedení zkonstruuje denotát výrazu. Jako příklad lze uvést
například výraz "francouzský král." V době psaní této práce Francie krále nemá. Výraz nemá žádný
denotát, neuvádí žádné konkrétní individuum. Přesto výrazu "francouzský král" rozumíme, výraz má
svůj význam, jen v současné době neuvádí žádnou osobu. A budeme-li chtít o významu výrazu
"francouzský král" něco vypovědět, například že francouzský král je monarchou v čele Francie,
daný monarcha nemusí existovat. Dále lze uvést například rozdíl mezi výrazy "logaritmus 1024
o základě 2" a "5 + 5". Denotátem obou výrazů je 10. Zadáme-li do interpreteru Haskellu výrazy
\lstset{language=Haskell}
\lstinline{logBase 2 1024} a \lstinline{5 + 5}, získáme v obou případech stejný výsledek.
V přirozeném jazyce ovšem chápeme značný rozdíl mezi oběma výrazy, ačkoliv mají stejný denotát.
"Logaritmus 1024 o základě 2" vyjadřuje číslo, kterým musíme umocnit dvojku, abychom získali 1024.
Výraz "5 + 5" očividně vyjadřuje úplně jinou matematickou operaci a jeho výsledek spočítáme jiným
postupem.


\begin{tikzpicture}
\end{tikzpicture}

\endinput
