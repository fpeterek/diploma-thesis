\chapter{Transparentní intenzionální logika}
\label{sec:TILIntroduction}

% TODO: Citace
Transparentní intenzionální logika (TIL) je logický systém založený na typovaném lambda kalkulu.
TIL je využíván k logické analýze přirozeného jazyka. Oproti tradičnímu lambda kalkulu, jenž
se využívá jako komputační model, tedy jako pouhý prostředek k dosažení konkrétní hodnoty --
výsledku, v Transparentní intenzionální logice hraje konstrukce kalkulu často důležitější roli,
než hodnota, kterou by konstrukce po provedení zkonstruovala.

Jako příklad využití lambda kalkulu jako výpočetní model lze uvést např. funkcionální programovací
jazyk Haskell. Interně je Haskell kompilován do lambda kalkulu (přesněji do jeho supersetu
obsahujícího např. čísla nebo logické hodnoty, která jinak v lambda kalkulu musíme kódovat pomocí
Churchova kódování -- K-kombinátorů, apod.). Ultimátně v Haskellu ovšem lambda kalkul slouží pouze
k získání konkrétního výsledku. Nadefinujeme vztah mezi vstupem a výstupem, a program napsaný
v Haskellu nám vstup transformuje. Pokud zanedbáme efektivitu programu, nezajímá nás, jakým
způsobem program spočítal výsledek, dokud jej spočítal správně.

Naopak Transparentní intenzionální logika je hyperintenzionálním kalkulem, který nám umožňuje
vytvářet konstrukce vypovídající o jiných konstrukcích. TIL vychází z myšlenky, že výraz
přirozeného jazyka sice označuje denotát -- konkrétní individuum, významem výrazu ovšem není
samotný denotát, který ani nemusí nutně existovat. Význam výrazu je abstraktní a lze jej zachytit
konstrukcí. Daná konstrukce poté při provedení zkonstruuje denotát výrazu. Jako příklad lze uvést
například výraz "francouzský král." V době psaní této práce Francie krále nemá. Výraz nemá žádný
denotát, neuvádí žádné konkrétní individuum. Přesto výrazu "francouzský král" rozumíme, výraz má
svůj význam, jen v současné době neuvádí žádnou osobu. A budeme-li chtít o významu výrazu
"francouzský král" něco vypovědět, například že francouzský král je monarchou v čele Francie,
daný monarcha nemusí existovat. Dále lze uvést například rozdíl mezi výrazy "logaritmus 1024
o základě 2" a "5 + 5". Denotátem obou výrazů je 10. Zadáme-li do interpreteru Haskellu výrazy
\lstset{language=Haskell}
\lstinline{logBase 2 1024} a \lstinline{5 + 5}, získáme v obou případech stejný výsledek.
V přirozeném jazyce ovšem chápeme značný rozdíl mezi oběma výrazy, ačkoliv mají stejný denotát.
"Logaritmus 1024 o základě 2" vyjadřuje číslo, kterým musíme umocnit dvojku, abychom získali 1024.
Výraz "5 + 5" očividně vyjadřuje úplně jinou matematickou operaci a jeho výsledek spočítáme jiným
postupem.

\begin{figure}
    \centering
    \begin{tikzpicture}
        \node[draw, fit={(0, 0) (2, 1)},              xshift=3cm, inner sep=0pt, label=center:Výraz] (A) {};
        \node[draw, fit={(0, 0) (2, 1)}, yshift=-5cm,             inner sep=0pt, label=center:Konstrukce] (B) {};
        \node[draw, fit={(0, 0) (2, 1)}, yshift=-5cm, xshift=6cm, inner sep=0pt, label=center:Denotát] (C) {};

        \path (A) -- node[sloped] (ab) {vyjadřuje}  (B);
        \path (A) -- node[sloped] (ac) {označuje}   (C);
        \path (B) -- node[sloped] (bc) {konstruuje} (C);

        \draw [-latex]          (A)--(ab)--(B);
        \draw [-latex] [dashed] (A)--(ac)--(C);
        \draw [-latex]          (B)--(bc)--(C);
    \end{tikzpicture}
    \caption{Schéma procedurální sémantiky TIL}
    \label{fig:til-semantics}
\end{figure}

Denotátem výrazu může být nejen objekt z báze, ale i konstrukce nebo funkce.

Jak již bylo zmíněno, Transparentní intenzionální logika vychází z typovaného lambda kalkulu, proto
také každý objekt musí mít svůj typ. Pro správné pochopení TILu, a tedy i této práce, je tak nutné 
znát typovou hierarchii TIL.

\section{Báze}

Báze je kolekce vzájemně disjunktních neprázdných množin, které dohromady definují s jakými objekty
budeme pracovat. Tyto množiny definují atomické objekty. Každá množina dále objektům určuje určitá
základní kritéria (např. pokud jako jednu z množin báze zvolíme množinu $\mathbb{N}$, víme, že
všechny objekty z této množiny budou čísla). Vždy se ovšem jedná pouze o nejnutnější
a nejzákladnější vlastnosti. Báze například nemá žádný vliv na vlastnosti proměnlivé v čase.

Bázi volíme dle potřeb konkrétní aplikace a univerza diskurzu. Například používáme-li TIL k logické
analýze matematických vět, jako bázi lze zvolit například množinu celých čísel, množinu reálných
čísel, a množinu pravdivostních hodnot. Musíme však vzít v potaz, že tato báze neobsahuje čísla
komplexní.

Patří-li objekt x do množiny $\alpha$ z báze, říkáme, že se jedná o objekt typu $\alpha$.
K explicitnímu uvedení typu objektu \textit{x} využíváme zápis $x/\alpha$. Množinám tvořícím bázi
lze přirozeně říkat typy.

Pro analýzu přirozeného jazyka se většinou volí objektová báze skládající se z typů {$o$, $\iota$,
$\tau$, $\omega$}. Tyto typy jsou podrobněji popsány v tabulce \ref{tab:default-base}.

\begin{table}
    \caption{Výchozí báze pro analýzu přirozeného jazyka}
    \label{tab:default-base}
    \centering

    \begin{tabular} { | l l | }
        \hline
        Typ      & Popis typu \\
        \hline
        $o$      & Množina pravdivostních hodnot \\
        $\iota$  & Množina individuí (univerzum diskurzu) \\
        $\tau$   & Množina časových okamžiků/reálných čísel \\
        $\omega$ & Množina logicky možných světů \\
        \hline
    \end{tabular}
\end{table}

\section{Funkce}

V matematice se jako základní molekulární typ využívají relace. Funkce je poté speciální typ relace,
který je zprava jednoznačný. V TIL je však základním molekulárním typem funkce. Chceme-li v TIL
vyjádřit $n$-ární relaci nad množinou $\alpha_1 \times ... \times \alpha_n$, lze tak samozřejmě
udělat definicí $n$-ární funkce z $\alpha_1 \times ... \times \alpha_n$ do $o$, která každému prvku
z $\alpha_1 \times ... \times \alpha_n$ přiřadí pravdivostní hodnotu na základě toho, zda prvek
do relace patří, nebo ne.

Narozdíl od tradičního lambda kalkulu, kde jsou funkce pouze nulární nebo unární, v Transparentní
intenzionální logice není arita funkce omezena.

\subsection{Intenze a extenze}

V TIL dále rozlišujeme funkce na tzv. \textit{intenze} a \textit{extenze}. Intenze jsou funkce
z možných světů. Extenze jsou funkce, jejichž doménou množina možných světů není, a tudíž jejich
funkční hodnota nezávisí na stavu světa.

Intenze jsou obecně funkce typu $(\alpha\omega)$ pro libovolný typ $\alpha$. Nejčastěji se však
jedná o funkce typu $((\alpha\tau)\omega)$, tedy funkce zobrazující možné světy do chronologií
objektů typu $\alpha$.

\section{Konstrukce TIL}

Konstrukce v Transparentní intenzionální logice jsou abstraktní procedury. Tyto procedury jsou
strukturované -- nejedná se o množiny, mají pevně danou strukturu, a na uspořádání případných
podprocedur záleží. Tyto konstrukce lze podle definovaných pravidel provést. Provedením konstrukce,
získáme výstup, případně nezískáme nic, viz pravidla provedení konstrukcí \ref{execution-rules}.

\section{Pravidla provedení konstrukcí} \label{execution-rules}

\subsection{Princip kompozicionality} 

\section{Typy 1. řádu}

Nechť \textit{B} je báze. Pak:

\begin{enumerate}[i)]
    \item Každá množina z báze \textit{B} je atomický typ řádu 1 nad \textit{B}.
    \item Nechť $\alpha, \beta_1, ...,\beta_m (m > 0)$ jsou typy řádu 1 nad \textit{B}. Pak soubor
        všech \textit{m}-árních parciálních funkcí $(\alpha\beta_1...\beta_m)$, tedy zobrazení z 
        $\beta_1 \times ... \times \beta_m$ do $\alpha$, je molekulární typ řádu 1 nad \textit{B}.
    \item Nic jiného není typem řádu 1 nad bází \textit{B}.
\end{enumerate}

\section{Rozvětvěná hierarchie typů}

\section{Charakteristické rysy TIL}

\endinput
