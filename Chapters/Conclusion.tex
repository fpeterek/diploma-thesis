\chapter{Závěr}

Cíl práce -- vytvořit funkční překladač jazyka TIL-Script, byl splněn. Byl implementován prototyp
překladače, který dokáže překládat programy jazyka TIL-Script, chybí mu ovšem některé v praxi
potřebné optimalizace (například optimalizace koncového volání). Nedostatky jsou však v textu
zdokumentovány, aby bylo možné překladač dále rozšiřovat a vylepšovat.

Dále práce rozšířila jazyk TIL-Script o nové prvky, jako jsou například definice nových typů,
komentáře, výrazy \lstinline{Import}, nebo textové řetězce.

Implementována byla také standardní knihovna pro základní práci s jazykem TIL-Script, nebo
matematická knihovna definující několik užitečných matematických funkcí. Standardní i matematická
knihovna jsou zdokumentovány v tomto textu.

Překladač umožňuje implementovat TIL-Script funkce, které interně volají funkce jazyka Java -- tím
se jazyku TIL-Script otvírá také celý Java ekosystém, včetně knihoven pro jazyk Java.

Při tvorbě překladače bylo myšleno také na budoucí rozvoj. Současnou implementaci překladač je možné
nahradit implementací novou. Pokud bude nový překladač implementovat potřebná rozhraní, bude tento
překladač plně kompatibilní se standardní a matematickou knihovnou, případně s jakoukoliv jinou
knihovnou psanou pro jazyk TIL-Script.

Nakonec byla práce také doplněna o řadu ukázek TIL-Script programů, ale také o ukázku tvorby
TIL-Script knihoven pomocí jazyků Java a Kotlin. Do budoucna lze jazyk TIL-Script i překladač dále
rozvíjet, kromě dříve zmíněných optimalizací by například bylo možné překladač doplnit o
interaktivní prostředí, které umožní zadávat TIL-Script věty a konstrukce interaktivně a postupně
je překládat. K interaktivnímu rozhraní by pak bylo možné vytvořit grafické prostředí, například
skrze webovou aplikaci. Přesto je však překladač funkčí a použitelný již v současné verzi, a snad
tedy umožní další vývoj na poli logické analýzy přirozeného jazyka.

\endinput
