\chapter{TILScript}

Nyní konečně přišel čas představit TILScript. TILScript je interpretovaný funkcionální programovací
jazyk, do znatelné míry inspirovaný jazyky jako Haskell nebo Lisp. Syntax TILScriptu by měla co
nejvíce připomínat syntaktické prvky Transparentní intenzionální logiky, aby pouhá znalost TILu
stačila k okamžitému pochopení TILScriptu. Sémantika by poté měla být stejná.

Tato kapitola je rozdělena do tří sekcí. V první sekci jsou popsány důležité základní rysy
TILScriptu. Druhá sekce popisuje již existující prvky TILScriptu, a případně dokumentuje změny
oproti předchozím verzím TILScriptu. Poslední sekce se věnuje navrhovaným rozšířením jazyka
TILScript.

\section{Charakteristické rysy TILScriptu}

Tato sekce popisuje charakteristické rysy TILScriptu v takové podobě, jakou nabývá v této práci.
Pokud se v některém bodě TILScript neshoduje s TILem či předchozími verzemi TILScriptu, je rozdíl
náležitě popsán a vysvětlen.

\subsection{Lambda kalkul parciálních funkcí}

\subsubsection{Shora neomezená arita funkcí}

% TODO: Ozdrojovat currying asi? Idk, lambda kalkul

Narozdíl od lambda kalkulu ve své tradiční podobě, nebo například jazyka Haskell, v Transparentní
intezionální logice není arita funkce shora omezená. TILScript musí tento fakt reflektovat. Proto
tento jazyk umožňuje definici i aplikaci funkcí libovolné (samozřejmě nezáporné) arity. Také zde
neexistuje rozvíjení funkcí (anglicky \textit{currying}). Zatímco např. v Haskellu jsou funkce
arity dvě nebo vyšší automaticky rozvinuty na sérií několika unárních funkcí, jejichž oborem hodnot
jsou unární nebo nulární funkce, a jedné nulární funkce která vrací žádaný výsledek, v TILScriptu
není arita nijak omezená.

\subsubsection{Parciální funkce a respektování principu kompozicionality}

Jelikož v TIL můžou být funkce parciální, musí i TILScript počítat s parcialitou funkcí. Dále musí
TILScript respektovat princip kompozicionality, základní rys Transparentní intenzionální logiky.
Jedním z důsledků principu kompozicionality je, že konstrukce, jejíž přinejmenším jeden konstituent
je nevlastní, bude také nutně nevlastní. Reprezentaci stavu, kdy parciální funkce je aplikována
na argumenty, na kterých není definována, se věnuje podsekce \textit{Hodnota Nil} \ref{nil-value}
této kapitoly. Způsob dodržování principu kompozicionality je popsán v podsekci
\textit{Okamžité vyhodnocování (Eager evaluation)} \ref{eager-eval}.

%TODO: Ocitovat
Jedinou výjimkou je funkce \lstinline{IsNil}, jež vrací pravdivostní hodnotu \lstinline{True},
pokud je její jediný argument \lstinline{Nil}, v opačném případě je jejím výsledkem
\lstinline{False}. Tato speciální sémantika funkce \lstinline{IsNil}, ačkoliv porušuje princip
kompozicionality a vyžaduje aplikaci unární funkce na "nic," je zvolena jako doplněk k funkci
\lstinline{Improper/(o*_n)} definované v Průvodci čtenáře, a jako kompromis mezi dodržením
principů TIL a umožněním zpracování chyb.

\subsection{Neměnnost proměnných a symbolů (\textit{Immutability})}

Jelikož je TILScript funkcionální jazyk, jsou hodnoty všech proměnných konstantní -- tedy
jakmile je proměnné jednou přiřazena valuace, nelze její hodnotu změnit. Dále nelze proměnnou
zastínit (angl. \textit{to shadow, shadowing}) v rámci oblasti platnosti (angl. \textit{scope}),
ve které byla definována. Proměnnou lze zastínit vytvořením nové oblasti platnosti (tedy například
na novém rámci zásobníku, angl. \textit{stack frame}).

% TODO: symboly

\subsection{Okamžité vyhodnocování (\textit{Eager evaluation})} \label{eager-eval}

\subsection{}

\section{TILScript jako výpočetní varianta TILu}

\section{Rozšíření TILScriptu}

\subsection{Hodnota \textit{Nil}} \label{nil-value}

\endinput
