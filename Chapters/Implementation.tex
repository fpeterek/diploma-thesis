\chapter{Implementace}

V této kapitole nastíníme implementační detaily interpreteru. Nejprve zmíníme využité technologie,
poté popíšeme architekturu projektu, nakonec pak uvedeme zajímavější problémy, které se objevily
při implementaci překladače, a jejich řešení.

\section{Zvolené technologie}

Celý projekt je implementován v jazyce \textbf{Kotlin} \footnote{Přestože je projekt psaný
  v Kotlinu, v práci často zmiňujeme např. \textit{Java knihovny}, \textit{Java objekty}, apod.
  Většinou tím myslíme Javu jako platformu. V takových případech poté nezáleží, zda
  používáme jazyk Java, Kotlin, nebo libovolný jiný jazyk kompilovaný pro platformu JVM.}.
Jazyk Kotlin je staticky typovaný, multiparadigmatický jazyk kompilovaný do \textbf{JVM} bytekódu.
Kotlin je vytvářen jako alternativa k jazyku Java, a nabízí plnou kompatibilitu s Javou. Využití
Kotlinu umožňuje využívat veškeré výhody Java ekosystému, včetně knihoven psaných v Javě, ale také
psát expresivnější kód, než by bylo možné v Javě. Null-safety a jazyková podpora pro algebraické
datové typy pak umožňují psát bezpečnešjí kód, než je možné v jazyce Java. Pro spuštění překladače
je samozřejmě nutné mít na počítači nainstalované JRE.

Jako build systém byl zvolen projekt \textbf{Gradle}. Důvodem této volby je relativní jednoduchost
konfigurace i využití systému Gradle, ale také přístup k Maven repozitáři s Java knihovnami. Dále
využíváme několik Gradle pluginů nutných k sestavení projektu.

Parser generujeme za pomoci technologie \textbf{Antlr} ve verzi 4. Antlr je open-source generátor
parserů podporující tvorbu parserů v řadě jazycích. V našem případě využíváme jako cílový jazyk
Javu.

Interpreter využívá knihovnu \lstinline{org.apache.commons:commons-text} pro zpracování escape
sekvencí. Tím výčet využitých technologií končí.

\section{Architektura projektu}

Celý projekt je rozdělen do čtyř komponent -- společné knihovny (\lstinline{common}), standardní
knihovny (\lstinline{stdlib}), překladače (\lstinline{interpreter}) a matematické knihovny
(\lstinline{math}). Schéma projektu je znázorněno v obrázku \ref{fig:project-structure}, který
vyjadřuje, jak na sobě jednotlivé komponenty projektu závisí.

\begin{figure}
    \centering
    \begin{tikzpicture}
        \node[draw, fit={(0, 0) (2, 1)},                xshift=3cm, inner sep=0pt, label=center:common] (A) {};
        \node[draw, fit={(0, 0) (2, 1)}, yshift=-2.5cm, xshift=3cm, inner sep=0pt, label=center:stdlib] (B) {};

        \node[draw, fit={(0, 0) (2, 1)}, yshift=-4cm,   xshift=6cm, inner sep=0pt, label=center:math] (C) {};
        \node[draw, fit={(0, 0) (2, 1)}, yshift=-4cm,   xshift=0cm, inner sep=0pt, label=center:interpreter] (D) {};

        \draw [-latex]          (A)--(B);
        \draw [-latex]          (A)--(C);
        \draw [-latex]          (A)--(D);
        \draw [-latex]          (B)--(C);
        \draw [-latex]          (B)--(D);
    \end{tikzpicture}
    \caption{Komponenty projektu}
    \label{fig:project-structure}
\end{figure}

\subsection{Společná knihovna \lstinline{common}}

Knihovna \lstinline{common} obsahuje kód společný pro zbytek projektu. Jedná se například
o implementace tříd reprezentujících TIL konstrukce, definice společných rozhraní, reprezentaci
typů, nebo drobné utility. Knihovna neobsahuje definice TIL-Script objektů, slouží ke sdílení kódu
napříč jednotlivými komponentami. Využít ji tak může například programátor implementující novou
TIL-Script knihovnu, konečného uživatele se však existence \lstinline{common} nijak netýká.

Knihovna nemá žádné externí závislosti.

\subsection{Standardní knihovna \lstinline{stdlib}}

Knihovna \lstinline{stdlib} obsahuje implementaci standardní knihovny. \lstinline{stdlib}
nekonformuje vůči rozhraní, kterému musí konformovat TIL-Script knihovny implementované jako Java
knihovny a distribuované jako Java archivy. Standardní knihovnu překladač automaticky importuje
v každém souboru. Není tedy třeba ji importovat explicitně.

Standardní knihovna je nezávislá na použitém překladači TIL-Scriptu. Interpreter, jenž je součástí
projektu, můžeme klidně nahradit novým interpreter (za předpokladu, že daný interpreter implementuje
potřebná rozhraní, např. \lstinline{InterpreterInterface} definované v knihovně \lstinline{common}).
Interpreter samotný však na standardní knihovně závisí. Kvůli syntaktickému cukru
(funkce \lstinline{Cond}, \lstinline{ListOf}, atd.), funkci \lstinline{If}, jež musí být
vyhodnocována líně, apod., musí překladač obsahovat speciální podporu pro standardní knihovnu.

Standardní knihovna definuje základní množinu funkcí, hodnot, typů a proměnných potřebnou pro práci
s TIL-Scriptem.

\subsection{Matematická knihovna \lstinline{math}}

Matematická knihovna \lstinline{math} slouží jako ukázková implementace TIL-Script knihovny
v Kotlinu, případně v Javě. Dále je využívána k testování funkčnosti importování Java knihoven.
Narozdíl od standardní knihovny, překladač je naprosto nezávislý na knihovně \lstinline{math}.
\lstinline{math} je třeba importovat explicitně pomocí výrazu \lstinline{Import}.

Knihovnu nelze označit za extenzivní, obsahuje pouze malé množství funkcí, definice symbolických
hodnot \lstinline{E}, \lstinline{Pi} a proměnných \lstinline{e}, \lstinline{pi} aproximujících
eulerovo číslo a číslo $\pi$.

\subsection{Interpreter}

Narozdíl od předchozích komponent, které byly čistě Java knihovnami, Interpreter je spustitelný Java
program (tedy program s korektně definovanou funkcí \lstinline{main}). Jedná se o referenční
implementaci překladače jazyka TIL-Script. Překladač podporuje TIL-Script v takové podobě, v jaké je
definován touto prací. Obsahuje základní nástroje pro hlášení chyb, aby ulehčil práci
s TIL-Scriptem. Typovou kontrolu provádí pouze za běhu programu.

\section{Implementace překladače}

V této sekci nejprve nastíníme fungování překladače. Poté popíšeme zajímavé problémy týkající se
implementace překladače. Problémy můžou být zajímavé jak z hlediska řešeného problému, tak
z hlediska budoucího rozvoje.

\subsection{Vysokoúrovňový pohled na interpreter}

Interpreter funguje čistě jako neinteraktivní textová aplikace. Programy, které potřebujeme
interpretovat, předáváme překladači jako CLI argumenty. Pro překladač momentálně neexistuje žádné
grafické rozhraní, ani REPL, který by umožnil interaktivní překlad TIL-Script výrazů. Pokud
využíváme funkci \lstinline{main} definovanou v JAR překladače, spustí se programy sekvenčně
v takovém pořadí, v jakém je uživatel specifikoval. Pokud ale běh jednoho z programu skončí chybou,
další programy se již nespouští. Pro každý program je však vytvořena nová instance překladače. Běhy
jednotlivých programů jsou tedy na sobě nezávislé.

Pro strojovou práci se zdrojovým kódem musíme nejdříve daný kód převést ze sekvenční textové podoby
do reprezentace, se kterou počítač umí pracovat lépe. Proto programy převádíme do stromové
reprezentace nazývané abstraktní syntaktický strom (také AST -- Abstract Syntax Tree). Tomuto převodu
se říká \textit{parsování}, případně anglicky \textit{parsing}. Parsing (včetně tokenizace a
lexikální analýzy) je v našem interpreteru delegován parseru automaticky vygenerovaném technologií
Antlr. Pokud lexer nebo parser narazí na syntaktické chyby v programu, jsou dané chyby ohlášeny
uživateli a překlad automaticky končí neúspěchem. Výsledkem vygenerovaného parseru je abstraktní
syntaktický strom.

Výsledné AST je ovšem pro naše potřeby příliš generické. Antlr je nástroj pro všeobecné použití,
proto i pomocí Antleru vytvořené AST bývají všeobecné. Antlr však kromě parseru umí vygenerovat také
abstraktní třídu \lstinline{Visitor} sloužící k průchodu AST pomocí návrhového vzoru
\textit{visitor}. Pomocí visitoru převedeme výsledek parsování na dočasnou reprezantaci --
mezivýsledek. Tento výsledek je, díky omezením automaticky vygenerované třídy, stále nedostatečný.
V průběhu tohoto průchodu interpreter nehledá chyby v programu -- k tomu chybí překladači v současném
momentě dostatečný kontext.

Proto proces parsingu následuje ještě jeden průchod stromem a jeho konečný převod na přívětivou
strukturu. Během tohoto průchodu překladač převede reprezentaci zdrojového kódu na objekty tříd
definovaných v knihovně \lstinline{common} -- tedy na standardní objekty očekávané napříč projektem
(překladačem, funkcemi, apod.). Dále v tomto průchodu překladač provede expanzi funkcí
\lstinline{Cond}, \lstinline{ListOf}, \lstinline{TupleOf}, \lstinline{Progn} z variadických funkcí,
jež v TILu neexistují, na sekvenci binárních funkcí. Během tohoto průchodu může interpreter opět
zachytávat chyby -- většinou se jedná o chybné využití syntaktického cukru

Následně již dochází k interpretaci programu. Jednotlivé výrazy ovšem nejsou interpretovány přesně
v takovém pořadí, v jakém jsou v programu uvedeny. Při překladu jsou nejprve provedeny výrazy
\lstinline{Import}. Pokud překladač narazí na výraz \lstinline{Import}, je interpretace aktuálního
souboru pozastavena a překladač přeloží importovaný soubor. Po přeložení importované závislosti
jsou do kontextu současného souboru naimportovány symboly definované závislostí \footnote{
  Symboly zde myslíme názvy -- jména funkcí, proměnných, i symbolických hodnot. V daném kontextu
  slovo symbol nesouvisí se symbolickými hodnotami, využíváme jej stejně jako jej například
  využívají linkery kompilovaných programovacích jazyků.
}.

Dále dochází k přednostní deklaraci struktur, kterou následuje interpretace výrazů
\lstinline{TypeDef}. Teprve po interpretaci \lstinline{TypeDef} dojde ke korektní definic struktur
a jejich atributů. Využitím tohoto postupu umožníme, aby výrazy \lstinline{TypeDef} mohly zmiňovat
struktury, a aby v definicích struktur bylo možné zmiňovat typové aliasy.

Poté je potřeba provést deklarace proměnných a symbolických hodnot (viz \ref{symbolic-values}),
a současně deklarace a definice funkcí. V tomto kroku jsou deklarace proměnných automaticky
vyvozeny z definic. Pokud program obsahuje definici proměnné, není nutné dodávat deklaraci. Dodání
nekonfliktní deklarace ovšem není chybou.

Nakonec jsou interpretovány top-level konstrukce a přiřazení proměnným. Pokud je konstrukce
na nejvyšší úrovni $v$-nevlastní, program končí chybou. Pokud však uživatel počítá s tím, že
konstrukce může být $v$-nevlastní, může její výsledek přiřadit proměnné, nebo jej zpracovat
pomocí funkce \lstinline{IsNil}.

Vždy, když je vyhodnocována určitá skupina výrazů, ať už při přednostním vyhodnocování, nebo při
vyhodnocování konstrukcí, pořadí výrazů v konkrétní skupině vždy odpovídá jejich relativnímu pořadí
ve zdrojovém programu.

\subsection{Detaily implementace}

Tato podsekce popisuje detaily implementace, které mohou být důležité nebo zajímavé. Cílem je
vyzdvihnout řešení, které jsou důležité z hlediska budoucího rozvoje, řešení zajímavých problémů,
ale také řešení, které jsou suboptimální a zasloužily by si další vývoj.

\subsubsection{Interpretace konstrukcí, správa paměti a operace nad AST}

Velkým designovým rozhodnutím, a zároveň nedostatkem, je rozhodnutí operovat přímo nad abstraktním
syntaktickým stromem a interpretovat přímo TIL-Script konstrukce reprezentované Java objekty. Kód
není překládán do bytekódu, který by byl následně interpretován.

Tento přístup k interpretaci má jednu velkou výhodu -- nízkou náročnost na implementaci. Překlad
do bytekódu by s sebou nesl značnou přidanou komplexitu. V první řadě by bylo potřeba implementovat
vlastní mechanismus zásobníku. V současné době TIL-Script sdílí zásobník s interpreterem. Aplikace
funkcí je řešena rekurzivně. Při aplikaci TIL-Script funkce volá interpreter interně vlastní metodu,
která aplikaci provede. S voláním metody na JVM je samozřejmě vytvořen nový rámec na zásobníku.
Na tomto rámci se poté uloží kontext aplikované TIL-Script funkce (argumenty funkce, apod.).
Po dokončení interpretace TIL-Script funkce se interpreter vrací z metody zpět na místo, odkud byla
metoda zavolána. S návratem je přirozeně spojena destrukce rámce na zásobníku -- destrukce, která
v současné implementaci obsahuje také zapomenutí kontextu funkce. Paměť využitá kontextem funkce
je poté uvolněna garbage collectorem během reklamačního cyklu.

Právě garbage collector je druhou výhodou současného přístupu. Interpretace AST a využití mechanismu
zásobníku z JVM nám umožňuje využít také garbage collector z JVM pro automatickou správu paměti.
Pokud bychom implementovali vlastní zásobník nezávislý na JVM, museli bychom naprogramovat také
vlastní garbage collector.

Ve své podstatě však mechanismus stacku ani mark-and-sweep algoritmus\footnote{Alespoň tedy
  mark-and-sweep algoritmus ve své nejjednodušší podobě. Mark-and-sweep algoritmus se používal již
  v padesátých letech minulého století, a používá se dodnes, v moderních jazycích je však silně
  optimalizován.} není složité naprogramovat.

Podstatně složitějším problémem by byla definice bytekódu, a také řešení obousměrného překladu mezi
bytekódem a JVM objekty. Ve většině programovacích jazyků stačí implementovat překlad AST
do interpretovatelného bytekódu. Dále poté překladač pouze interpretuje daný bytekód. TIL-Script
ovšem umožňuje definovat funkce nejen pomocí konstrukcí Transparentní intenzionální logiky, ale také
jako Java funkce. Tyto funkce psané v Javě přirozeně jako své arugmenty očekávají JVM objekty.
Pokud bychom však překládali zdrojový kód do bytekódu, museli bychom před voláním Java funkce
převést všechny její argumenty z bytekódu zpět na JVM objekty.

V závislosti na definici bytekódu a implementaci interpreteru by také mohlo být třeba implementovat
vlastní haldu a alokátor paměti, neboť by nemuselo být možné využít haldu a alokátory JVM, které
využívá současná implementace.

Největší nevýhodou současné implementace ovšem je chybějící optimalizace koncového volání (TCO).
Optimalizací koncového volání rozumíme opakované využití jednoho rámce zásobníku pro více
rekurzivních volání funkce. Tradičně při musíme při volání funkce vytvořit nový rámec na zásobníku.
Moderní překladače ovšem dokážou v určitých situacích rekurzi zoptimalizovat, využít stejný rámec
pro více po sobě jdoucích volání, a efektivně tak nahradit rekurzi cyklem. Tuto optimalizaci lze
provést pouze pokud k rekurzivnímu volání dojde při opouštění funkce (proto \textit{koncové}
volání). Ačkoliv TCO implementuje také řada překladačů imperativních jazyků (např. GCC, Clang),
nejčastěji se s touto optimalizací setkáváme právě ve funkcionálních jazycích, ve kterých nelze
implementovat cyklus, a proto jsme odkázání čistě na rekurzi. V jazyku bez optimalizace koncové
rekurze bohužel dojde velmi rychle k přetečení zásobníku.\footnote{
  Zásobník JVM lze zvětšit pomocí vlajky \lstinline{-Xss}. Zvětšením zásobníku problém s jeho
  přetékáním můžeme částečně omezit.
}

Z časových omezení se bohužel optimalizaci koncové rekurze, ani vše, co by s její implementací bylo
nutně spojeno (vlastní zásobník, správu paměti, překladač a interpreter bytekódu), nepodařilo
implementovat. Čas byl věnován především rozvoji jazyka TIL-Script a možností jeho interpretace.
Implementaci TCO je ovšem třeba do budoucna zvážit, a tato sekce tak může posloužit alespoň jako
návod, jak pokračovat v rozvoji TIL-Scriptu.

\subsubsection{Java objekty jako TIL-Script funkce}

TIL-Script nabízí uživatelům dva způsoby definice funkce. Prvním způsobem je abstrakce
nad konstrukcí, ať už pomocí uzávěru, nebo pomocí syntaxe pro definici pojmenované funkce. Druhým
způsobem je již dříve zmiňovaná definice funkce pomocí Java objektu.

Aby byl JVM objekt použitelný jako funkce jazyka TIL-Script, musí být daný objekt instancí třídy
\lstinline{DefaultFunction}. Daný objekt poté musí definovat jméno a typ funkce, a současně metodu
\lstinline{apply} volanou právě při aplikaci dané funkce. Argumenty této metody jsou instance
\lstinline{InterpreterInterface}, představující aktuálně používaný interpreter, seznam všech
argumentů funkce, a také kontext aplikace funkce. Kontext aplikace obsahuje pozici ve zdrojovém
kódu, kde k aplikaci funkce došlo, a slouží především k hlášení chyb. Návratovou hodnotou metody
\lstinline{apply} je výsledek aplikace funkce na dané argumenty.

Metoda \lstinline{apply} může vrátit \lstinline{Nil}, nikdy však \lstinline{Nil} nemůže obdržet
jako jeden ze svých argumentů -- pokud by alespoň jeden argument měl hodnotu \lstinline{Nil},
k aplikaci funkce, a tedy ani volání metody \lstinline{apply}, nikdy nedojde.

Definice TIL-Script funkcí pomocí Java objektů umožňuje využívat v jazyce TIL-Script již existující
Java knihovny. Například místo implementace volání funkcí systému (tzv. \textit{syscall}),
a následné implementace komunikace po síti přímo nad operačním systémem v jazyce TIL-Script lze
využít již existující knihovny napsané v jazyce Java. Pro komunikaci s databází by tedy stačilo
využít již napsanou knihovnu pro platformu JVM, a pouze nad ní vytvořit jednoduché TIL-Script
rozhraní.

Takto reprezentovanou funkci poté stačí importovat v TIL-Script programu za využití registrátoru.
Registrátory jsou popsány hned v následující podsekci \peteref{java-import}.

\subsubsection{Import symbolů z Java archivu}\label{java-import}

Výhody umožnění implementace TIL-Script funkcí pomocí Java objektů již byly popsány. Výrazy
\lstinline{Import} byly popsány v sekci \peteref{import-statement}. Nyní se budeme věnovat způsobu,
jakým jsou symboly importovány z Java archivů.

Abychom vůbec mohli importovat definice z Java archivů, musí být daný JAR soubor načtený již při
spouštění JRE. Java runtime ve snaze zabránit kyberútokům (například RCE útokům) značně ztěžuje
načítání kódu za běhu. Způsoby neomezeného načítání kódu existují, liší se však napříč verzemi
platformy Java, ale hlavně představuje potenciální velký bezpečnostní problém.

Abychom mohli bezpečně načíst kód za běhu, museli bychom znát nejen název třídy, jež potřebujeme
načíst, ale také cestu k JAR souboru, ve kterém se třída nachází. Pokud bychom po uživateli jazyka
TIL-Script chtěli, ať uvádí i Java archiv i plně kvalifikovaný název registrátoru, byly by
\lstinline{Import} výrazy příliš složité. Uvést pouze Java archiv nestačí, neboť nemusíme znát obsah
daného archivu. Abychom mohli uvést pouze registrátor, musíme znát JAR soubor dopředu. Poslední
verze se zdá pro konečného uživatele jazyka TIL-Script nejjednodušší.

Zvolený způsob importu Java objektů dále obsahuje ještě jednu výhodu -- umožňuje definovat více
registrátorů v jednom JAR souboru, ale podle potřeby načítat a využívat pouze konkrétní registrátor.
Není tedy třeba importovat více názvů, než je nezbytně nutné.

Nyní popíšeme implementaci samotného vytvoření instance registrátoru a importu symbolů. Překladač
interpretuje výrazy \lstinline{Import} přednostně hned při spuštění programu. Narazí-li na výraz
\lstinline{Import}. Nejprve zkontroluje, zda parametr výrazu obsahuje předponu \lstinline{class://}.
Pokud ne, překladač automaticky považuje daný parametr za cestu k TIL-Script souboru (nezávisle
na příponě, konvencí je však používat příponu \textit{.tils}). Pokud parametr začíná požadovanou
přeponou, překladač přeponu odstraní a pokusí se načíst specifikanou třídu. Následně vytvoří
instanci této třídy pomocí konstruktoru bez parametrů (všechny registrátory tedy musí tento
konstruktor definovat). Dále již překladač s vytvořenou instancí pracuje jako s objektem typu
\lstinline{SymbolRegistrar}. Všechny registrátory proto musí toto rozhraní implementovat.

% TODO: Continue


\subsubsection{Typová kontrola}

\subsubsection{Mechanismus zásobníku}

\subsubsection{Uzávěry a $\lambda$-vázané proměnné}

\subsubsection{Nezávislost knihoven na překladači}

Projekt je koncipován tak, aby byly TIL-Script knihovny nezávislé na intepreteru, který uživatel
použije. Implementovaný překladač je plně funkční, díky časovým omezením ovšem překladači chybí
některé nutné prvky, jako třeba optimalizace koncového volání, jež je ve funkcionálních
jazycích nezbytná, či překlad do bytekódu umožňující efektivnější překlad. V budoucnu se může stát,
že bude potřeba nahradit současný překladač efektivnější implementací. V takovém případě ovšem není
žádoucí, aby nastala potřeba přepsat nebo upravit také všechny již existující TIL-Script knihovny,
standardní knihovnu, apod. Pro implementaci TIL-Script knihovny je však potřeba, aby např.
TIL-Script funkce implementované v Javě měly určitý přístup k interpreteru, nebo alespoň kontextu,
ve kterém jsou volány. Jinak by tyto funkce nemohly přistupovat např. k proměnným, k jiným funkcím,
apod.

Za účelem odstranění závislosti na konkrétním překladači definuje knihovna \lstinline{common}
rozhraní \lstinline{InterpreterInterface}. Toto rozhraní specifikuje základní funkce, které
překladač jazyka TIL-Script musí implementovat. Při volání TIL-Script funkcí definovaných pomocí
Java objektů je danému objektu předána instance překladače jako potomek
\lstinline{InterpreterInterface}.

Tento princip je v softwarovém inženýrství velmi známý, je však nutné, aby na modularitu systému
bylo myšleno již od počátku a aby byl dobře zdokumentován. Tato práce umožňuje, aby bylo v budoucnu
možno vytvořit drop-in náhradu za konkrétní překladač bez jakékoliv úpravy existujícího kódu.

\endinput
