\chapter{Implementace}

V této kapitole nastíníme implementační detaily interpreteru. Nejprve zmíníme
využité technologie, poté popíšeme architekturu projektu, nakonec pak uvedeme
zajímavější problémy, které se objevily při implementaci překladače, a jejich
řešení.

\section{Zvolené technologie}

Celý projekt je implementován v jazyce \textbf{Kotlin}. Jazyk Kotlin je staticky
typovaný, multiparadigmatický jazyk kompilovaný do \textbf{JVM} bytekódu.
Kotlin je vytvářen jako alternativa k jazyku Java, a nabízí plnou kompatibilitu
s Javou. Využití Kotlinu umožňuje využívat veškeré výhody Java ekosystému,
včetně knihoven psaných v Javě, ale také psát expresivnější kód, než by bylo
možné v Javě. Null-safety a jazyková podpora pro algebraické datové typy pak
umožňují psát bezpečnešjí kód, než je možné v jazyce Java. Pro spuštění
překladače je samozřejmě nutné mít na počítači nainstalované JRE.

Jako build systém byl zvolen projekt \textbf{Gradle}. Důvodem této volby je
relativní jednoduchost konfigurace i využití systému Gradle, ale také přístup
k Maven repozitáři s Java knihovnami. Dále využíváme několik Gradle pluginů
nutných k sestavení projektu.

Parser generujeme za pomoci technologie \textbf{Antlr} ve verzi 4. Antlr je
open-source generátor parserů podporující tvorbu parserů v řadě jazycích.
V našem případě využíváme jako cílový jazyk Javu.

Interpreter využívá knihovnu \lstinline{org.apache.commons:commons-text}
pro zpracování escape sekvencí. Tím výčet využitých technologií končí.

\section{Architektura projektu}

Celý projekt je rozdělen do čtyř komponent -- společné knihovny
(\lstinline{common}), standardní knihovny (\lstinline{stdlib}), překladače
(\lstinline{interpreter}) a matematické knihovny (\lstinline{math}). Schéma
projektu je znázorněno v obrázku \ref{fig:project-structure}, který vyjadřuje,
jak na sobě jednotlivé komponenty projektu závisí.

\begin{figure}
    \centering
    \begin{tikzpicture}
        \node[draw, fit={(0, 0) (2, 1)},                xshift=3cm, inner sep=0pt, label=center:common] (A) {};
        \node[draw, fit={(0, 0) (2, 1)}, yshift=-2.5cm, xshift=3cm, inner sep=0pt, label=center:stdlib] (B) {};

        \node[draw, fit={(0, 0) (2, 1)}, yshift=-4cm,   xshift=6cm, inner sep=0pt, label=center:math] (C) {};
        \node[draw, fit={(0, 0) (2, 1)}, yshift=-4cm,   xshift=0cm, inner sep=0pt, label=center:interpreter] (D) {};

        \draw [-latex]          (A)--(B);
        \draw [-latex]          (A)--(C);
        \draw [-latex]          (A)--(D);
        \draw [-latex]          (B)--(C);
        \draw [-latex]          (B)--(D);
    \end{tikzpicture}
    \caption{Komponenty projektu}
    \label{fig:project-structure}
\end{figure}

\subsection{Společná knihovna \lstinline{common}}

Knihovna \lstinline{common} obsahuje kód společný pro zbytek projektu. Jedná
se například o implementace tříd reprezentujících TIL konstrukce, definice
společných rozhraní, reprezentaci typů, nebo drobné utility. Knihovna neobsahuje
definice TIL-Script objektů, slouží ke sdílení kódu napříč jednotlivými
komponentami. Využít ji tak může například programátor implementující novou
TIL-Script knihovnu, konečného uživatele se však existence \lstinline{common}
nijak netýká.

Knihovna nemá žádné externí závislosti.

\subsection{Standardní knihovna \lstinline{stdlib}}

Knihovna \lstinline{stdlib} obsahuje implementaci standardní knihovny.
\lstinline{stdlib} nekonformuje vůči rozhraní, kterému musí konformovat
TIL-Script knihovny implementované jako Java knihovny a distribuované jako Java
archivy. Standardní knihovnu překladač automaticky importuje v každém souboru.
Není tedy třeba ji importovat explicitně.

Standardní knihovna je nezávislá na použitém překladači TIL-Scriptu.
Interpreter, jenž je součástí projektu, můžeme klidně nahradit novým interpreter
(za předpokladu, že daný interpreter implementuje potřebná rozhraní, např.
\lstinline{InterpreterInterface} definované v knihovně \lstinline{common}).
Interpreter samotný však na standardní knihovně závisí. Kvůli syntaktickému
cukru (funkce \lstinline{Cond}, \lstinline{ListOf}, atd.), funkci
\lstinline{If}, jež musí být vyhodnocována líně, apod., musí překladač obsahovat
speciální podporu pro standardní knihovnu.

Standardní knihovna definuje základní množinu funkcí, hodnot, typů a proměnných
potřebnou pro práci s TIL-Scriptem.

\subsection{Matematická knihovna \lstinline{math}}

Matematická knihovna \lstinline{math} slouží jako ukázková implementace
TIL-Script knihovny v Kotlinu, případně v Javě. Dále je využívána k testování
funkčnosti importování Java knihoven. Narozdíl od standardní knihovny, překladač
je naprosto nezávislý na knihovně \lstinline{math}. \lstinline{math} je třeba
importovat explicitně pomocí výrazu \lstinline{Import}.

Knihovnu nelze označit za extenzivní, obsahuje pouze malé množství funkcí,
definice symbolických hodnot \lstinline{E}, \lstinline{Pi} a proměnných
\lstinline{e}, \lstinline{pi} aproximujících eulerovo číslo a číslo $\pi$.

\subsection{Interpreter}

Narozdíl od předchozích komponent, které byly čistě Java knihovnami, Interpreter
je spustitelný Java program (tedy program s korektně definovanou funkcí
\lstinline{main}). Jedná se o referenční implementaci překladače jazyka
TIL-Script. Překladač podporuje TIL-Script v takové podobě, v jaké je definován
touto prací. Obsahuje základní nástroje pro hlášení chyb, aby ulehčil práci
s TIL-Scriptem. Typovou kontrolu provádí pouze za běhu programu.

\section{Implementace překladače}

\subsubsection{Nezávislost knihoven na překladači}

Projekt je koncipován tak, aby byly TIL-Script knihovny nezávislé na intepreteru,
který uživatel použije. Implementovaný překladač je plně funkční, díky časovým
omezením ovšem překladači chybí některé nutné prvky, jako třeba optimalizace
koncového volání (TCO), jež je ve funkcionálních jazycích nezbytná, či překlad
do bytekódu umožňující efektivnější překlad. V budoucnu se může stát, že bude
potřeba nahradit současný překladač efektivnější implementací. V takovém případě
ovšem není žádoucí, aby nastala potřeba přepsat nebo upravit také všechny již
existující TIL-Script knihovny, standardní knihovnu, apod. Pro implementaci
TIL-Script knihovny je však potřeba, aby např. TIL-Script funkce implementované
v Javě měly určitý přístup k interpreteru, nebo alespoň kontextu, ve kterém jsou
volány. Jinak by tyto funkce nemohly přistupovat např. k proměnným, k jiným
funkcím, apod.

\endinput
