\chapter{Implementace}

V této kapitole nastíníme implementační detaily interpreteru. Nejprve zmíníme
využité technologie, poté popíšeme architekturu projektu, nakonec pak uvedeme
zajímavější problémy, které se objevily při implementaci překladače, a jejich
řešení.

\section{Zvolené technologie}

Celý projekt je implementován v jazyce \textbf{Kotlin}. Jazyk Kotlin je staticky
typovaný, multiparadigmatický jazyk kompilovaný do \textbf{JVM} bytekódu.
Kotlin je vytvářen jako alternativa k jazyku Java, a nabízí plnou kompatibilitu
s Javou. Využití Kotlinu umožňuje využívat veškeré výhody Java ekosystému,
včetně knihoven psaných v Javě, ale také psát expresivnější kód, než by bylo
možné v Javě. Pro spuštění překladače je samozřejmě nutné mít na počítači
nainstalované JRE.

Jako build systém byl zvolen projekt \textbf{Gradle}. Důvodem této volby je
relativní jednoduchost konfigurace i využití systému Gradle, ale také přístup
k Maven repozitáři s Java knihovnami. Dále využíváme několik Gradle pluginů
nutných k sestavení projektu.

Parser generujeme za pomoci technologie \textbf{Antlr} ve verzi 4. Antlr je
open-source generátor parserů podporující tvorbu parserů v řadě jazycích.
V našem případě využíváme jako cílový jazyk Javu.

Interpreter využívá knihovnu \lstinline{org.apache.commons:commons-text}
pro zpracování escape sekvencí. Tím výčet využitých technologií končí.

\section{Architektura projektu}

Celý projekt je rozdělen do čtyř komponent -- společné knihovny
(\lstinline{common}), standardní knihovny (\lstinline{stdlib}), překladače
(\lstinline{interpreter}) a matematické knihovny (\lstinline{math}). Schéma
projektu je znázorněno v obrázku \ref{fig:project-structure}, který vyjadřuje,
jak na sobě jednotlivé komponenty projektu závisí.

\begin{figure}
    \centering
    \begin{tikzpicture}
        \node[draw, fit={(0, 0) (2, 1)},              xshift=3cm, inner sep=0pt, label=center:common] (A) {};
        \node[draw, fit={(0, 0) (2, 1)}, yshift=-2.5cm, xshift=3cm, inner sep=0pt, label=center:stdlib] (B) {};

        \node[draw, fit={(0, 0) (2, 1)}, yshift=-4cm, xshift=6cm, inner sep=0pt, label=center:math] (C) {};
        \node[draw, fit={(0, 0) (2, 1)}, yshift=-4cm, xshift=0cm, inner sep=0pt, label=center:interpreter] (D) {};

        \draw [-latex]          (A)--(B);
        \draw [-latex]          (A)--(C);
        \draw [-latex]          (A)--(D);
        \draw [-latex]          (B)--(C);
        \draw [-latex]          (B)--(D);
    \end{tikzpicture}
    \caption{Komponenty projektu}
    \label{fig:project-structure}
\end{figure}

\subsection{Společná knihovna \lstinline{common}}

Knihovna \lstinline{common} obsahuje kód společný pro zbytek projektu. Jedná
se například o implementace tříd reprezentujících TIL konstrukce, definice
společných rozhraní, reprezentaci typů, nebo drobné utility. Knihovna nemá
žádné externí závislosti.

\subsection{Standardní knihovna \lstinline{stdlib}}

\endinput
